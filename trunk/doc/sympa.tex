%
% Copyright (C) 1999, Institut Pasteur & Christophe Wolfhugel
% Copyright (C) 1999, Comit� R�seau des Universit�s & Serge Aumont, Olivier Sala�n
% Copyright (C) 1999, UVSQ & Pierre David
% 
%
% Historique
%   1999/04/12 : pda@prism.uvsq.fr : conversion to latex2e
%


\documentclass [twoside,a4paper] {report}

    \usepackage {epsfig}
    \usepackage {xspace}
    \usepackage {makeidx}
    \usepackage {html}

    \usepackage {palatino}
    \renewcommand {\ttdefault} {cmtt}

    \setlength {\parskip} {5mm}
    \setlength {\parindent} {0mm}

    \pagestyle {headings}
    \makeindex

    \sloppy

    \usepackage [dvips] {changebar}
    % \begin {changebar} ... \end {changebar}
    % ou \cbstart ... \cbend   et \cbdelete

    %
    % Change bars are not well rendered by latex2html
    %

    \begin {htmlonly}
        \renewcommand {\cbstart} {}
        \renewcommand {\cbend} {}
        \renewcommand {\cbdelete} {}
    \end {htmlonly}

    % black text on a white background, links unread in red
    % \bodytext {TEXT="#000000" BGCOLOR="#ffffff" LINK="#ff0000"}
    % black text on a white background
    \bodytext {TEXT="#000000" BGCOLOR="#ffffff"}

    \newcommand {\fig} [2]
    {
        \begin {figure} [htbp]
            \hrule
            \vspace {3mm}
            \begin {center}
                \epsfig {figure=#1.ps}
%                \epsffile {figure=#1.ps}
            \end {center}
            \vspace {2mm}
            \caption {#2}
            \vspace {3mm}
            \hrule
            \label {fig:#1}
        \end {figure}
    }

    \newcommand {\version} {2.7}

    \newcommand {\samplelist} {mylist}

    % #1 = text to index and to display
    \newcommand {\textindex} [1] {\index{#1}#1}

    % #1 = sort key, #2 displayed in text and index
    \newcommand {\textindexbis} [2] {\index{#1@#2}#2}

    \newcommand {\Sympa} {\textit {Sympa}\xspace}

    \newcommand {\WWSympa} {\textindexbis {WWSympa}{\textit {WWSympa}}\xspace}

    % #1 = sort key, #2 : displayed in text and index, #3 displayed in index
    \newcommand {\ttindex} [3]  {\index{#1@\texttt {#2} #3}\texttt {#2}}

    \newcommand {\example} [1] {Example: \texttt {#1}}

    \newcommand {\unixcmd} [1] {\ttindex {#1} {#1} {Unix command}}

    \newcommand {\mailcmd} [1] {\ttindex {#1} {#1} {mail command}}

    \newcommand {\cfkeyword} [1] {\ttindex {#1} {#1} {configuration keyword}}

    \newcommand {\default} [1]  {(Default value: \texttt {#1})}

    \newcommand {\scenarized} [1] {\texttt {#1} parameter is defined by scenario (see~\ref {scenarii}, page~\pageref {scenarii})}

    \newcommand {\lparam} [1] {\ttindex {#1} {#1} {list parameter}}

    \newcommand {\file} [1] {\ttindex {#1} {#1} {file}}

    \newcommand {\dir} [1]  {\ttindex {#1} {#1} {directory}}

    \newcommand {\tildefile} [1] {\ttindex {#1} {\~{}#1} {file}}

    \newcommand {\tildedir} [1] {\ttindex {#1} {\~{}#1} {directory}}

    \newcommand {\rfcheader} [1] {\ttindex {#1:} {#1:} {header}}

    % Notice: use {\at} when using \mailaddr
    \newcommand {\at} {\char64}
    \newcommand {\mailaddr} [1] {\texttt {#1}}   
% mail address
%        {\ttindex {#1} {#1} {mail address}}


\begin {document}

    \title {\Huge\bf Sympa \\ \huge\bf Mailing Lists Management Software}
    \author {
        Serge Aumont,
        Olivier Sala\"un,
        Christophe Wolfhugel,
         }
    \date {September 2000}
\begin {htmlonly}
For printing purpose, use the 
\htmladdnormallink {postscript format version} {sympa.ps} of this documentation.
\end {htmlonly}

\maketitle


{
    \setlength {\parskip} {0cm}



    \cleardoublepage

    \tableofcontents
    % \listoffigures
    % \listoftables
}

%%%%%%%%%%%%%%%%%%%%%%%%%%%%%%%%%%%%%%%%%%%%%%%%%%%%%%%%%%%%%%%%%%%%%%%%%%%%%%
% Presentation
%%%%%%%%%%%%%%%%%%%%%%%%%%%%%%%%%%%%%%%%%%%%%%%%%%%%%%%%%%%%%%%%%%%%%%%%%%%%%%

\cleardoublepage
\chapter {Presentation}

\Sympa is an electronic mailing list manager.  It is used to automate
list management functions such as subscription, moderation,
archive and shared documents management. 
It also includes management functions which
would normally require a high quantity of work that may be time
consumming and costly for the list owner.  Examples of these
functions include automatic management of subscription renewals,
list maintenance ...

\Sympa is designed to manage many different kind of lists. It includes
a web interface for all list function including management. It allows
fine definition of each list feature such as sender autorisation,
moderating process,  etc. \Sympa is designed to define for each list
and for each feature on the list who can perform the
operation and which method is used to authenticate this personn. Currently
the authentication available method can be based either on the smtp From
header, a password, or S/MIME signature.\\
\Sympa is also able extract electronic
addresses from an LDAP directory or any SQL server and include them
dynamically in a list.

\Sympa  manages sending of messages to the lists, e.g.  their
routing up to the destination, and therefore makes it possible to
reduce the load on the computer system hosting \Sympa.  In
configurations that have enough memory, \Sympa is especially well
adapted for large lists: for a 20,000 subscribers list, it takes
less than 6 minutes to send a message to 95 percent of the subscribers,
assuming that the network is available (tested on a 300~MHz, 256~MB
i386 serveur with Linux).

This guide covers installation, configuration and management of
the current release (\version) of
\htmladdnormallink {sympa} {http://listes.cru.fr/sympa/}.

\section {License}

\Sympa is a free software, you may distribute it under the terms
of the
\htmladdnormallinkfoot {GNU General Public License Version 2}
        {http://www.gnu.org/copyleft/gpl.html}

You may make and give away verbatim copies of the source form of
this package without restriction, provided that you duplicate all
of the original copyright notices and associated disclaimers.

\section {Features}

\Sympa provide all the basic features that any mailing list management robot
should include. Most \Sympa features have equivalent into different
software, but \Sympa provides all of thoses features
into a single software, including:

\begin {itemize}
    \item High speed distribution process and load control. \Sympa
        can be tuned to allow the system administrator to control
        the amount of computer resources used.  Its algorithm
        is optimized:

        \begin {itemize}
            \item use of your prefered SMTP engine, e.g.
                \unixcmd {sendmail}, \unixcmd {qmail} or \unixcmd
                {postfix}.

            \item tune the maximum number of SMTP child processes.

            \item sort email by domain and tune recipient grouping
                factor (depending on repartion of subscriber's
                domain)

            \item detailed logs.

        \end {itemize}

    \item \textbf {Multilingual} messages. Current version of
        \Sympa allows the administrator to choose the language
        catalog at run time. Currently, \Sympa robot is available for
        french, english, spanish, italian, german, finish and chinese
	(Big5 and GB) languages (web interface is available
        for french, english and spanish).

    \item \textbf {MIME support}. Of course, \Sympa respects
        \textindex {MIME} in the distribution process but also,
        \Sympa allows list owner to configure their list with
        welcome, goodbye and other predefined messages using complex
        \textindex {MIME} structure, e.g. welcome message can be
        \textbf {multipart/alternative} using \textbf {text/html},
        \textbf {audio/x-wav}~:-) or whatever (Note that \Sympa
        recognize commands in multipart messages as soon as one
        part is \textbf {text/plain }).

    \item The \textbf {sending process is controlled} per list.
        The list definition allows many different actions for
        each incomming message. A \lparam {private} list is a list where
        only subscribers can send messages. A list configured using
        \lparam {privateoreditorkey} mode accepts incoming messages
        from subscribers, otherwise \Sympa forwards it to the editor
        with a one time secret numeric key that will be used by the
        editor to \textit {reject} or \textit {distribute} the
        message. For details about all sending modes, refer to the
        \lparam {send} parameter (\ref {par-send}, page~\pageref
        {par-send}). The sending process configuration (as well as most other list
	operation) is defined using  a \textbf {scenario}. Any listmaster
        can define new scenario in order to describe some configuration
        different from the 20 predefined configuration distributed. \\
        Exemple : forward multipart messages to the list editor and  immediatly
        distribute the others.
        
    \item Privileged operations can be performed by list editor or
        list owner (or any other user category) as defined in list
        \file {config} file or by
        the robot \textindex {administrator}: listmaster defined
        in the \file {/etc/sympa.conf} global configuration file.
        Those operation are the usual \mailcmd {ADD}, \mailcmd
        {DELETE} or \mailcmd {REVIEW} commands which can be
        authentified using a one time password or using S/MIME signature.
	 Any list owner using \mailcmd {EXPIRE}
        command can ask for renewal of subscriptions.  This is made
        possible thanks to the subscription date stored in the
        \Sympa database.


    \item Dynamically include electronic addresses from a source
        that can be either a database accepting SQL queries or an
        LDAP directory. \Sympa keeps in his cache the data source
        using a TTL (Time To Live) parameter for reasonable response
        delays.

    \item Include lists subscriber set in one list. This is a real
        inclusion, not a dirty subscribtion a lists to a list.

    \item Internal subscriber data structure can be stored in a
        database or, for compatibility with versions 1.x, in text
        files. The choice of using a database was driven by the
        \WWSympa project.  The database ensures a secure access to
        shared data. The perl database API \textit {dbi/dbd} enables
        interoperability with various RDBMS (MySQL, PostgreSQL,
        Oracle, Sybase).

    \label {wwsympa} 
    \item {\WWSympa} is a global Web interface to all \Sympa functions
    	(including administration). It supplies :

        \begin {itemize}

	    \item classification of lists, search index

            \item access controle to all functions including to the list of lists
                  so WWsympa is particulary adpated to be the main groupware tool
                  of a intranet
 
       	    \item management of list shared documents (download, upload, specific
		  access control for each document)

            \item to each user, an HTML document presenting his current
                  subscriptions including access to archives, subscription options

            \item to list owners, list management tools (bounce processing, edition of
                  list parameters, moderation of incoming messages

            \item to robot administrators, list creation, global robot
                  configuration \index{administrator}

        \end {itemize}



\end {itemize}


\section {Project directions}

Future developpement should introduce:

\begin {itemize}

    \item Virtual robot definition, ISP would appreciate the
        equivalent of \htmladdnormallinkfoot {Apache} {http://www.apache.org}
        virtual server feature applied to mailing lists.

    \item Additional groupeware features such as calendar, scenari editor...

    \item Pluggin commercial antivirus
 
    \item Subscriber preference for multipart/alternative messages reception.
          Some news paper publish both text/plain and text/html version of
          there mails. It should be nice to provide a sungle list for both format.

\end {itemize}

\section {History}

\Sympa developement started from scratch in 1995. The goal was to
ensure continuity for the \textindex {TULP} list manager, produced
partly by the initial author of \Sympa: Christophe Wolfhugel.

New functionalities  were required, but for which the code of TULP
was no more adequate to continue its life cycle. \Sympa initial
version brings authentication, flexible management of commands,
high performances in internal data access and object oriented code
for easy code maintenance.

It took nearly two years to come out  with the first market releases.

\section {Authors and credits}

Christophe Wolfhugel is the author of the first beta version of
\Sympa, this work was performed while he was working for the
\htmladdnormallinkfoot {Institut Pasteur} {http://www.pasteur.fr}.

Later developments are mainly driven by the
\htmladdnormallinkfoot {Comit\'e R\'eseaux des Universit\'es} {http://www.cru.fr}
(Olivier Sala\"un and Serge Aumont) in charge of a large mailing
list service.

We would like to thank contributors, among which:

\begin {itemize}

   \item Pierre David, who in addition to his help and suggestions
       in developing the code, participated more than actively in
       the manual that you have in your hands.

  \item Ollivier Robert, Usenet Canal Historique and the good manners
      guru in the Perl program.

  \item Rapha\"el Hertzog (debian) and St\'ephane Poirey (redhat) for
      Linux packages.

  \item Olivier Lacroix for all his perseverance in bug fixing.

  \item Fabien Marquois who introduced many new features such as
      digest.

  \item Alex Nappa and Josep Roman for their spanish translations

  \item Carsten Clasohm and Jens-Uwe Gaspar for their german translations

  \item Marco Ferrante for his italian translations

  \item Hubert Ulliac for search in archive base on marcsearch.pm

  \item Tung Siu Fai for his chinese translations

  \item and also: Manuel Valente, Dominique ROUSSEAU,
    Laurent Ghys, Francois Petillon, Guy Brand, Jean Brange, Fabrice
    Gaillard, Herv� Maza

   \item Persuaded anonymous ones who never missed a chance to
       remind us that \textit {smartlist} already did all that
       better.

   \item All contributors and beta-testers cited in the \file
       {RELEASE\_NOTES} file who by serving as guinea pigs and
       being the first to use it, made it possible to quickly and
       efficiently debug the \Sympa software.

    \item Bernard Barbier, without whom \Sympa would not
        have a name.

\end {itemize}

May those people we have forgotten to thank accept our apologies
and make us aware of it so that we can correct this error in future
releases of this documentation.

\section {Mailing list and support}
    \label {sympa@cru.fr}

If you wish to contact the authors of \Sympa, please use the address
\mailaddr {sympa-authors{\at}cru.fr}.

There are also a \htmladdnormallinkfoot {few mailing-lists about \Sympa} {http://listes.cru.fr/wws/lists/informatique/sympa} :

	\begin {itemize}
	   \item  \mailaddr {sympa-users{\at}cru.fr} general info list
	   
	   \item   \mailaddr {sympa-fr{\at}cru.fr}, for french-speaking users
			   
	   \item   \mailaddr {sympa-announce{\at}cru.fr}, \Sympa announces
			  
	   \item   \mailaddr {sympa-dev{\at}cru.fr}, \Sympa developpers
			  
	\end {itemize}

To join, send the following message to \mailaddr {sympa{\at}cru.fr}:

\begin {quote}
    \texttt {subscribe} \textit {Listname} \textit {Firstname} \textit {Name}
\end {quote}

(replace \textit {Listname}, \textit {Firstname} and \textit {Name} by the list name, your first name and your family name).

You may also have a look at the \Sympa \htmladdnormallink {home page} {http://listes.cru.fr/sympa},
you will find the upto date version, \htmladdnormallink {FAQ} {http://listes.cru.fr/sympa/fom-serve/cache/1.html} and so on.

%%%%%%%%%%%%%%%%%%%%%%%%%%%%%%%%%%%%%%%%%%%%%%%%%%%%%%%%%%%%%%%%%%%%%%%%%%%%%%
% Overview, what is it made of ?
%%%%%%%%%%%%%%%%%%%%%%%%%%%%%%%%%%%%%%%%%%%%%%%%%%%%%%%%%%%%%%%%%%%%%%%%%%%%%%
\cleardoublepage
\chapter {Overview, what is it made of ?}

%\begin {htmlonly}
%<A NAME="overview">
%\end {htmlonly}

\section {Organization}
\label {organization}

Here is a snapshot of what \Sympa looks like once it has settled 
on your system. This also reveals the pruduct's philosophy I guess.

\begin {itemize}

	\item \tildedir {sympa/}\\
	The root directory of \Sympa, you will find almost everything
	related to \Sympa under this directory, except logs and main
	configuration files.
	
	\item \tildedir {sympa/bin/}\\
	This directory contains the binaries, including CGI. It
	also contains the default scenarios, templates and configuration
	files.

	\item \tildedir {sympa/bin/etc/}\\
	\Sympa will find here defaults for what it expects in 
	\tildedir {sympa/etc/} (scenarios, templates and configuration
	files, recognized S/Mime certificates).

	\item \tildedir {sympa/etc/}\\
	This is your site's configuration directory. Take a look at
	\tildedir {sympa/bin/etc/} for designing your own.

	\item \tildedir {sympa/etc/create\_list\_templates/}\\
	Here are your list's templates, those proposed at list
	creation time.

	\item \tildedir {sympa/etc/scenari/}\\
	This directory will contain your scenarios. If you don't
	know what the hell is a scenario, refer to \ref {scenarii}, 
	page~\pageref {scenarii}.
	
	\item \tildedir {sympa/etc/wws\_templates/}\\
	The web interface (\WWSympa) is made of template HTML
	files parsed by the CGI program.

	\item \tildedir {sympa/etc/templates/}\\
	Some replies of the mail robot are defined by templates
	(\file{welcome.tpl} for SUBSCRIBE). You can overload
	these template files in the list's directory, here are
	the defaults.

	\item \tildedir {sympa/expl/}\\
	The working directory of \Sympa.

	\item \tildedir {sympa/expl/\samplelist}\\
	The list directory, refer to \ref {list-directory}, 
	page~\pageref {list-directory}.

	\item \tildedir {sympa/nls/}\\
	Internationalization directory. It contains XPG4-compatible
	messages catalogues. \Sympa has currently been translated
	to 8 different languages.

	\item \tildedir {sympa/spool/}\\
	\Sympa uses 6 different spools.

	\item \tildedir {sympa/src/}\\
	Here is where we store \Sympa sources.

\end {itemize}

\section {Binaries}
\label {binaries}

\begin {itemize}

	\item \file {sympa.pl}\\
	The main daemon ; it processes commands and delivers
	messages. It is permanently scanning the \dir {msg/} spool.

	\item \file {wwsympa.fcgi}\\
	This is the CGI program, proposing a complete web interface
	to mailing lists. It can work in both classical CGI and
	FastCGI modes ; we recommand FastCGI mode though, being
	up to 10 times faster.

	\item \file {bounced.pl}\\
	This daemon processes bounces (non-delivery messages),
	looking for the bad addresses. List owners will later
	access bounces information via \WWSympa. It is permanently 
	scanning the \dir {bounce/} spool.

	\item \file {archived.pl}\\
	This daemon feeds the web archives, converting messages
	to HTML format and linking them. It uses the amazing 
	\file {MhOnArc}. It is permanently scanning the \dir {outgoing/} 
	spool.

	\item \file {queue}\\
	This small program gets the incoming messages from the aliases
	and store them in \dir {msg/} spool.

	\item \file {bouncequeue}\\
	Same as \file {queue} for bounces. It stores bounces in 
	\dir {bounce/} spool.

\end {itemize}

\section {Configuration files}

\begin {itemize}

	\item \file {sympa.conf}\\
	The main configuration file.
	See \ref{exp-admin}, page~\pageref{exp-admin}.
	

	\item \file {wwsympa.conf}\\
	\WWSympa configuration file.
	See \ref{wwsympa}, page~\pageref{wwsympa}.
	
	\item \file {edit\_list.conf}\\
	It defines what parameters/files are editable by
	owners. See \ref{list-edition}, page~\pageref{list-edition}.

	\item \file {topics.conf}\\
	Define your site's topics (classification in \WWSympa) in this 
	file with titles. A sample is provided in \dir {sample/} directory
	of the sympa distribution.
	See \ref{topics}, page~\pageref{topics}.

\end {itemize}

\section {Spools}
\label {spools}

\begin {itemize}

	\item \tildedir {sympa/spool/auth/}\\
	For storing messsages until it has been confirmed.

	\item \tildedir {sympa/spool/bounce/}\\
	For storing incoming bouncing messages.

	\item \tildedir {sympa/spool/digest/}\\
	For storing lists' digests before it is sent.

	\item \tildedir {sympa/spool/expire/}\\
	Related to expire process.

	\item \tildedir {sympa/spool/mod/}\\
	For storing unmoderated messages.

	\item \tildedir {sympa/spool/msg/}\\
	For storing incoming messages (including commands).

	\item \tildedir {sympa/spool/outgoing/}\\
	\file {sympa.pl} dumps messages i this spool 
	for \file {archived.pl} to archive them.

\end {itemize}


%%%%%%%%%%%%%%%%%%%%%%%%%%%%%%%%%%%%%%%%%%%%%%%%%%%%%%%%%%%%%%%%%%%%%%%%%%%%%%
% Installing Sympa
%%%%%%%%%%%%%%%%%%%%%%%%%%%%%%%%%%%%%%%%%%%%%%%%%%%%%%%%%%%%%%%%%%%%%%%%%%%%%%
\cleardoublepage
\chapter {Installing \Sympa}

%\begin {htmlonly}
%<A NAME="installsympa">
%\end {htmlonly}

\Sympa is a program written in Perl language. It also calls a short
program written in C to perform some tasks for which it is not
reasonable to entrust an interpreted language.

\section {Getting \Sympa, related links}

The \Sympa distribution is available from
\htmladdnormallink {\texttt {http://listes.cru.fr/sympa/}}
    {http://listes.cru.fr/sympa/}.
All important resources are referenced there:

\begin {itemize}
    \item sources
    \item \file {RELEASE\_NOTES}
    \item .rpm and .deb packages for Linux
    \item user mailing list
        (see~\ref {sympa@cru.fr}, page~\pageref {sympa@cru.fr})
    \item contributions
    \item ...
\end {itemize}


\section {Prerequisites}

\Sympa installation and configuration are relatively easy
tasks for an administrator who already has a good deal of experience
with Unix systems and installation of tools written in Perl.

The user will note that a large part of the installation time will
be devoted to installation of the prerequisites if they are not
already on the system. The number of ancillary tools required is
not really of great importance, but on a current Unix system, it
should be possible to do this without much of a problem. We strongly
advise carrying out the system installations and checks in the
order listed below; these steps will be explained in detail in later
sections.

\begin {itemize}
    \item identification of characteristics of Unix system that
        will host \Sympa.

    \item installation of DB Berkeley module (already installed on
      most UNIX systems).

    \item installation of
        \htmladdnormallinkfoot {CPAN (Comprehensive Perl Archive Network)}
                {http://www.perl.com/CPAN}
        modules.

    \item creation of a Unix user.

\end {itemize}

\subsection {System requirements}

You should have a Unix system that is more or less recent in order
to be able to use \Sympa. In particular, it is necessary
that your system have an ANSI C compiler (on the whole, in the case
which is of interest to us, your compiler should support prototypes)
as well as XPG4-standard \textindex {NLS} (Native Language Support, for
languages other than English) extensions.

\Sympa has been installed and tested on the following
systems, therefore you should not have any special problems:

\begin {itemize}
    \item Linux (various distributions)
    \item FreeBSD 2.2.x and 3.x
    \item Digital Unix 4.x
    \item Solaris 2.5 and 2.6
    \item AIX 4.x
    \item HP-UX 10.20
\end {itemize}

Anyone willing to port it on NT ? ;-)

If your Unix system has a \unixcmd {gencat} command as well as
\unixcmd {catgets(3)} and \unixcmd {catopen(3)} functions, it is
likely that it has \textindex {NLS} extensions and that these extensions comply
with the XPG4 specifications.

Finally, most Unix systems are now supplied with an ANSI C compiler;
if this is not the case, you can install the \unixcmd {gcc} compiler,
which you will find on the nearest GNU site, for example
\htmladdnormallinkfoot {in France} {ftp://ftp.oleane.net/pub/mirrors/gnu/}.

To complete the operation, you should make sure that you have a
recent enough release of the \unixcmd {sendmail} MTA, e.g. release
\htmladdnormallinkfoot {8.9.x} {ftp://ftp.oleane.net/pub/mirrors/sendmail-ucb/}
or a more recent release. You may also use \unixcmd {postfix} or
\unixcmd {qmail}.

\subsection {Install Berkeley DB (NEWDB)}

Unix systems often have a highly simple system to manage indexed
files.  This consists of extensions known under the name \texttt
{dbm} and \texttt {ndbm}. For many programs, these are not sufficient,
and in particular for \Sympa, which uses the \textindex {DB package}
initially developed at the University of California in Berkeley,
and which is now maintained by the company \htmladdnormallinkfoot
{Sleepycat software} {http://www.sleepycat.com}.  Many UNIX  systems
like Linux, FreeBSD or Digital Unix 4.x have the DB package in the
standard version. If not you should install this tool if you have not 
already done so.

You can retrieve DB on the
\htmladdnormallinkfoot {Sleepycat site} {http://www.sleepycat.com/}
and read its documentation to install it in your system.

\subsection {Install Perl and CPAN modules}

To be able to use \Sympa, you must have release 5.004\_03 of the
Perl language or a later release and several CPAN modules.

At \texttt {make} time, \unixcmd {check\_perl\_modules.pl} script is run to
check required Perl and CPAN modules versions. If a CPAN module is
missing or out of date, this script will install it for you. 

You can also download and install CPAN modules yourself. You will find 
a current release of the Perl interpreter in the nearest CPAN archive. 
If you do not know where to find a nearby site, use the
\htmladdnormallinkfoot {CPAN multiplexer} {http://www.perl.com/CPAN/src/latest.tar.gz};
it will find one for you.

\subsection {Required CPAN modules}

Here are CPAN modules required by \Sympa that not incuded in Perl's
distribution. We'll try to keep this list uptodate ; if you have any doubt
run \unixcmd {check\_perl\_modules.pl} script.

\begin {itemize}
   \item DB\_File (v. 1.50 or higher)
   \item Msgcat
   \item MD5
   \item MailTools (version 1.13 and upper)
   \item MIME-tools (may require IO/Stringy)
   \item MIME-Base64
\end {itemize}

Since release 2, \Sympa requires a DMBS to work properly. It stores 
users subscriptions and preferences in a database. \Sympa is also
able to extract user data from within an external database. 
These features require that you install database-related Perl libraries.
This includes the generic Database interface (DBI) and a Database Driver
for your RDBMS (DBD) :

\begin {itemize}
   \item \textbf {DBI} (DataBase Interface)

   \item \textbf {DBD} (DataBase Driver) related to your RDBMS (e.g.
       Msql-Mysql-modules for MySQL)

\end {itemize}

If you plan to interface \Sympa with an LDAP directory to build
dynamical mailing lists, you need to install Perl LDAP libraries :

\begin {itemize}
    \item \textbf {Net::LDAP} (perlldap).

\end {itemize}

\subsection {Create a Unix user}

Last step before installing \Sympa: create a Unix user, and
preferably, a group specific to the program. Most of the installation
will be carried out with this account. We suggest that you use the
name \texttt {sympa} for both user and group. Note that UID.GID must be
the same as your httpd. If you are running a dedicated httpd server,
this can be sympa.sympa, else it is possible either to define a virtual httpd
server setting UID GID, either to run \Sympa as nobody.nobody. 
This last solution is not safe because the information managed by \Sympa
will be owned by nobody.

In the rest of this documentation, numerous files will be located
in the user's login directory which you will have created in your
system or in a sub-directory. We will mark this login directory
with \tildedir {sympa/}.

\section {Compilation and installation }

Before using \Sympa, you have to customize the sources so that you
can specify a few specific parameters in your installation.

To start, extract the sources from the archive file, for example
from the \tildedir {sympa/} directory: the archive will create a
directory named \dir {sympa-\version/} where all the useful files
and directories will be located. In particular, you will have a
\dir {doc/} directory containing this documentation in various
formats, a \dir {sample/} directory containing a few examples of
a configuration file, \dir {nls/} directory where multi-linguage
messages are stored and of course, the \dir {src/} directory for the
mail robot and \dir {wwsympa} for the web interface.

Example:

\begin {quote}
\tt
\# su - \\
\$ gzip -dc sympa-\version.tar.gz | tar xf -
\end {quote}

Before running \unixcmd {make} in the main
directory, you have to edit and configure \file {Makefile}
so that you can customize the first part
in them. We advise against changing anything located after the STOP
line.

The \file {Makefile} file contains explanations for the fields,
which you may have to change. They are mainly :
\begin {itemize}
\item USER and GROUP, the id of daemons.
\item CONFIG and WWSCONFIG, the robot and the cgi configuration location
\item DIR, the \Sympa home dir
\item MAILERPROGDIR, the location of queue and bouncequeue programs. If sendmail
is configured to used smrsh (check the mailer prog definition in your sendmail.cf), queue and bouncequeue needs to be installed in /etc/smrsh.  This is probably your case if you are using
redhat 6.X.
\item INITDIR, the directory to install SYSV init script (typically /etc.rc.d/init.d/)
\item DESTDIR, can be set in the main Makefile to install sympa in DESTDIR/DIR (instead of DIR).
This is usefull for building RPM and DEB packages.
\item PERL, SH and CC and GENCAT, respectivly perl, sh, cc and gencat location. 
\item DARK\_COLOR, LIGHT\_COLOR, TEXT\_COLOR, BG\_COLOR, ERROR\_COLOR to define wwsympa
RBG colors
\end {itemize}

Once this file has been configured, you need to run \texttt
{make;make~install} commands.  This
 generates the binary  for the \file {queue} program, and the nls and installs
the \Sympa and \WWSympa programs in their final slot, while having propagated
a few parameters in the Perl files, as for example the access path
to the Perl program. The make command includes a checking of cpan modules.
 
If everything goes smoothly, the \tildedir {sympa/bin/} directory
will contain various Perl programs as well as the \file {queue}
binary.  You will note that this binary has the \index{set-uid-on-exec
bit} \textit {set-uid-on-exec} bit (owner is the \texttt {sympa}
user): this is normal and indispensable for \Sympa to run correctly.

\subsection {Choosing directory location}

All directories are defined in \file {/etc/sympa.conf} file which
is read by \Sympa at runtime. While installing \Sympa, if no \file {sympa.conf} file
was found, a sample one will be installed.
To see the default directories organization refer to \ref {organization}, 
page~\pageref {organization}.

It is possible to disperse files and directories everywhere in the
Unix system, however we recommand storing all the directories and
files in  the \texttt {sympa} user's login directory.

You also have to create these directories now. You can use restrictive
authorizations if you like, since only programs running with the
\texttt {sympa} account will have to access them.


\section {Robot aliases}
    \index{aliases}
    \index{mail aliases}

An electronic list manager such as \Sympa runs in two steps:

\begin {itemize}
    \item a message sent to a list or to \Sympa itself
        (for subscribe, unsubscribe, help messages, etc.) is received
        by the SMTP server (\unixcmd {sendmail} or \unixcmd {qmail}).
        The SMTP server, on reception of this message, runs the
        \file {queue} program (supplied in this package) to store
        the message in a queue, e.g.  in a special directory.

    \item the \file {sympa.pl} daemon, run once and for all during
        startup of the system, scans the queue. As soon as it
        detects a new message, it processes it and performs the
        requested action (distribution or processing of an
        administrative request).

\end {itemize}

To differentiate processing of administrative requests (subscription,
unsubscription, help request, etc.) from messages sent to a list,
a special mail alias is reserved for administrative requests, so
that \Sympa can be accessible and users can join it. Thus, the
lines below must be added to the \unixcmd {sendmail} alias file
(often \file {/etc/aliases}):

\begin {quote}
\begin{verbatim}
sympa:             "| /home/sympa/bin/queue sympa"
listmaster: 	   "| /home/sympa/bin/queue listmaster"
bounce+*:          "| /home/sympa/bin/bouncequeue sympa"
sympa-request:     postmaster
sympa-owner:       postmaster
\end{verbatim}
\end {quote}

\mailaddr {sympa-request} should be the address of the robot
\textindex {administrator}, who is the person who takes care of
\Sympa management (here \mailaddr {postmaster{\at}cru.fr}).

\mailaddr {sympa-owner} is the return address for \Sympa error
messages.

The alias bounce+* is dedicated to collect bounces. It is usefull
only if at least one list use \texttt { welcome\_return\_path unique } or
\texttt { remind\_return\_path unique}.
Do not forget to run \unixcmd {newaliases} after any change to
the \file {/etc/aliases} file!

Note: aliases based on \mailaddr {listserv} (in addition to those
based on \mailaddr {sympa}) can be added in order to facilitate
the task of users accustomed to the \mailaddr {listserv} and
\mailaddr {majordomo} names. For example:

\begin {quote}
\begin{verbatim}
listserv:          sympa
listserv-request:  sympa-request
majordomo:         sympa
listserv-owner:    sympa-owner
\end{verbatim}
\end {quote}

Note: it will also be necessary to add entries in this alias file
to create lists (see list creation section, \ref {list-aliases},
page~\pageref {list-aliases}).


\section {Logs}

\Sympa keeps traces of each of its procedures in its log file.
Nevertheless, this requires configuration of the \unixcmd {syslogd}
daemon.  \Sympa's by default is to use \texttt {local1} facility
(\lparam {syslog} parameter in \file {sympa.conf})
,WWSympa uses (\lparam {log\_facility} parameter
in \file {wwsympa.conf}), the default is \Sympa's log facility.\\
Thus, a line must be added in the \unixcmd {syslogd} configuration file (\file
{/etc/syslog.conf}). For example:

\begin {quote}
\begin{verbatim}
local1.*       /var/log/sympa 
\end{verbatim}
\end {quote}

Then reload \unixcmd {syslogd}.

Depending on your platform, your syslog daemon may use either
UDP or Unix socket. \Sympa's default is to use Unix socket ;
you may change this behaviour by editing \file {sympa.conf}'s
\lparam {log\_socket\_type} parameter (\ref{par-log-socket-type},
page~\pageref{par-log-socket-type}).

\section {Daemon loading}

Once all files are configured, all you have to do is start \Sympa.
At startup, \file {sympa.pl} will change its UID to sympa (as defined in \file {Makefile}).
To do this, add the following sequence or its equivalent in your
\file {/etc/rc.local}:

\begin {quote}
\begin{verbatim}

~sympa/bin/sympa.pl
~sympa/bin/archived.pl
~sympa/bin/bounced.pl

\end{verbatim}
\end {quote}

\file {sympa.pl} recognises the following command line arguments:

\begin {itemize}

\item -d 
  
  Sets \Sympa in debug mode and keeps it attached to the terminal. 
  Debug information are printed to STDERR as well as standard log
  information.
  
\item -dD 

  Sets debug mode in maximum verbosity, tracing each subroutine calls. Usefull for
  contributor.
  
\item -f \textit {config\_file}
  
  Forces \Sympa to use an alternate configuration file. Default behaviour is
  to use configuration file as defined in the Makefile (\$CONFIG).
  
\item -m 
  
  \Sympa will log calls to sendmail, including recipients. Usefull to
  keep track of each mail sent (log files may grow faster though).
  
\item -l \textit {catalog}
  
  Set this option to use a language catalog for \Sympa. 
  The corresponding catalog file must be located in \tildedir {sympa/nls}
  directory. 
  
  For example, with the \file {fr.cat} catalog:
  
  \begin {quote}
\begin{verbatim}
/home/sympa/bin/sympa.pl
\end{verbatim}
  \end {quote}
  
\end {itemize}


%%%%%%%%%%%%%%%%%%%%%%%%%%%%%%%%%%%%%%%%%%%%%%%%%%%%%%%%%%%%%%%%%%%%%%%%%%%%%%
% sympa.conf params
%%%%%%%%%%%%%%%%%%%%%%%%%%%%%%%%%%%%%%%%%%%%%%%%%%%%%%%%%%%%%%%%%%%%%%%%%%%%%%
\cleardoublepage
\chapter {sympa.conf parameters}
    \label {exp-admin}
    \index{sympa.conf}
    \index{configuration file}

The \file {/etc/sympa.conf} configuration file contains numerous
parameters which are read on start-up of \Sympa and allow the
program to run.  If you have to change this file, do not forget
that you have to restart \Sympa. 

The \file {/etc/sympa.conf} file contains directives in the following
format:

\begin {quote}
    \textit {keyword    value}
\end{quote}

Comments start with the \texttt {\#} character at the beginning of
a line.  Empty lines are also considered as comments and are ignored.
There should only be one directive per line, but their order in
the file is of no importance.

\section {Site customization}

\subsection {\cfkeyword {host}}

        This keyword is \textbf {mandatory}. It is the name of the
        machine used in \rfcheader {From} header during responses
        to administrative requests.

        \example {host listhost.cru.fr}

\subsection {\cfkeyword {email}} 
	
	\default {sympa}

        Address (left part, name before the \texttt {@} sign) used
        in \rfcheader {From} header during responses to administrative
        requests.

        \example {email           listserv}

\subsection {\cfkeyword {listmaster}} 

        The list of email address  of listmasters (person who can perform
        global  serveur command).

        \example {listmaster postmaster@cru.fr,root@home.cru.fr}

\subsection {\cfkeyword {wwsympa\_url}}  

	 \default {http://<host>/wws}

	This is the root URL of \WWSympa.

        \example {wwsympa\_url https://my.server/wws}

\subsection {\cfkeyword {cookie}} 

	This string is used to generate MD5 authentication keys.
	It allows such authentication keys to differ from one
	site to another. It is also used for reversible encryption of
        user password stored in database. Therefore you need to restrict
        \file {sympa.conf} access to Sympa user. 
       
        Don't forget that changing this parameter will break all
        http cookies stored in user's browser and all user password.

        \example {cookie gh869jku5}

\section {Directories}

\subsection {\cfkeyword {home}}

	 \default {\tildedir {sympa/expl}}

        The directory containing lists' subdirectories.

        \example {home          /home/sympa/expl}

\subsection {\cfkeyword {etc}}

	 \default {\tildedir {sympa/etc}}

        This is the local directory for configuration files (like
	\file {edit\_list.conf}. It contains 3 subdirectories :
	\dir {scenari} for local scenari and \dir {templates}
	for local site's templates and default lists' templates and \dir {wws\_templates}
        for local site's html templates.

        \example {home          /home/sympa/etc}

\section {System related}

\subsection {\cfkeyword {syslog}} 

	\default {LOCAL1}

        Name of sub-system (facility) for sending messages.

        \example {syslog          LOCAL2}

\subsection {\cfkeyword {log\_socket\_type}} 
    \label {par-log-socket-type}

	\default {inet}

        \Sympa communication with \unixcmd {syslogd} can be performed
        using either UDP or Unix socket type.  Set \cfkeyword
        {log\_socket\_type} to \texttt {inet} to use UDP or \texttt
        {unix} for Unix sockets.

\subsection {\cfkeyword {pidfile}} 

	\default {\tildefile {sympa/sympa.pid}}

        The file where the \file {sympa.pl} daemon stores its
        process number. Warning: the \texttt {sympa} user must be
        able to write in this file, and to create it if it doesn't
        exist.

        \example {pidfile         /var/run/sympa.pid}

\subsection {\cfkeyword {umask}} 

	\default {027}

        Default mask for file creation (see \unixcmd {umask}(2)).

        \example {umask 007}

\section {Sending related}

\subsection {\cfkeyword {maxsmtp}} 

	\default {20}

        Maximum number of SMTP delivery child processes launched
        by  \Sympa. This is the main load control parameter.

        \example {maxsmtp           500}


\subsection {\cfkeyword {max\_size}} 

	\default {5 Mb}

	Maximum size allowed for messages distributed by \Sympa.
	This may be custumized per list by setting \lparam {max\_size} 
	list parameter.

        \example {max\_size           2097152}

\subsection {\cfkeyword {nrcpt}} 

	\default {25}

        Maximum number of recipients per \unixcmd {sendmail} call.
        This grouping factor makes it possible for the MTA (\unixcmd
        {sendmail}) to optimize the number of SMTP sessions for
        message distribution.

\subsection {\cfkeyword {avg}} 

	\default {10}

        Maximum number of different internet domains within addresses per
        \unixcmd {sendmail} call.

\subsection {\cfkeyword {sendmail}} 

	\default {/usr/sbin/sendmail}

        Absolute call path to SMTP message transfert agent (\unixcmd
        {sendmail} for example).

        \example {sendmail        /usr/sbin/sendmail}


\subsection {\cfkeyword {rfc2369\_header\_fields}} 

	\default {help,subscribe,unsubscribe,post,owner,archive}

	RFC2369 compliant header fields (List-xxx) to be added to 
	distributed messages. These header-fields should be implemented
	by MUAs, adding menus.

\subsection {\cfkeyword {remove\_headers}} 

        \default {Return-Receipt-To,Precedence,X-Sequence,Disposition-Notification-To}

        This is the list of headers that \Sympa should remove from the
        outgoing messages. Use it, for example, to ensure some privacy
        to your users, while not retaining all the anonymous options.
        It is (for the moment) site-wide. It is applied before the
        \Sympa, {rfc2369\_header\_fields}, and {custom\_header} fields are
        added.

\example {remove\_headers      Resent-Date,Resent-From,Resent-To,Resent-Message-Id,Sender,Delivered-To,Return-Receipt-To,Precedence,X-Sequence,Disposition-Notification-To}


\section {Spool related}

\subsection {\cfkeyword {spool}}

        \default {\tildedir {sympa/spool}}

	The parent directory which contains all the other spools.  
        

\subsection {\cfkeyword {queue}} 

        The absolute path of the directory which contains the queue, used both by the
        \file {queue} program and the \file {sympa.pl} daemon. This
        parameter is mandatory.

        \example {queue          /home/sympa/queue}


\subsection {\cfkeyword {queuemod}}  
        \label {cf:queuemod}
        \index{moderation}

	\default {\tildedir {sympa/spool/moderation}}

        This parameter is optional and still supported for backward compatibility.


\subsection {\cfkeyword {queuedigest}}  
        \index{digest}
        \index{spool}

	\default {\tildedir  {digest}}

        This parameter is optional and still supported for backward compatibility.

\subsection {\cfkeyword {queueexpire}}  

	\default {\tildedir {sympa/spool/expire}}

        This parameter is optional and still supported for backward compatibility.

\subsection {\cfkeyword {queueauth}} 

	\default {\tildedir {sympa/spool/auth}}

        This parameter is optional and still supported for backward compatibility.

\subsection {\cfkeyword {queueoutgoing}} 

	\default {\tildedir {sympa/spool/outgoing}}

	This parameter is optional and still supported for backward compatibility.

\subsection {\cfkeyword {queuebounce}} 
    \index{bounce}

	\default {\tildedir {sympa/spool/bounce}}

        Spool to store bounces (non-delivery reports) received at \samplelist-owner 
	or bounce+* address by \file {bouncequeue} program. This parameter is mandatory
        and must be a absolute path.


\subsection {\cfkeyword {sleep}}  
        \label {kw-sleep}

	\default {5}

        Waiting period (in seconds) between each scan of main queue.
        Never set this value to~0!

\subsection {\cfkeyword {clean\_delay\_queue}} 

	\default {1}

        Conservation time (in days) for ``bad'' messages in
        \textindex {spool} (as specified by \cfkeyword {queue}).
        \Sympa keeps messages rejected for various reasons (badly
        formatted messages, messages coming from loops, etc.) in
        this directory, under a name prefixed by \texttt {BAD}.
        This configuration variable controls the number of days
        these messages are kept.

        \example {clean\_delay\_queue 3}

\subsection {\cfkeyword {clean\_delay\_queuemod}} 

	\default {10}

        Expiration date (in days) in \textindex {moderation}
        \textindex {spool} (as specified by \cfkeyword {queuemod}).
        Beyond this deadline, messages that have not been processed
        are deleted.  The content of this spool can be consulted
        for each moderated list using a key along with the \mailcmd
        {MODINDEX} command.

\subsection {\cfkeyword {clean\_delay\_queueauth}}  

	\default {3}

        Expiration delay (in days) in an \textindex {authentication}
        queue.  Beyond this deadline, messages not enabled are
        deleted.

\section {Internationalization related}    

\subsection {\cfkeyword {msgcat}}   

	\default{\tildedir {sympa/nls}}

        This is the multilinguale catalogs location (nls). Must be
        set accordind to the \tildefile {src/nls/Makefile}.

\subsection {\cfkeyword {lang}}   

	\default {us}

        The is the language for \Sympa. The corresponding message
	catalog (.cat) located in \cfkeyword {msgcat} directory
	will be used.

\section {Bounce related}

\subsection {\cfkeyword {bounce\_warn\_rate}}
        \label {kw-bounce-warn-rate}
         
        \default {30}

	This is the site default value for \lparam {bounce}.
	Above this rate of bouncing addresses in a list, listowner
	receives a warning for each mail distributed in the
	list.

\subsection {\cfkeyword {bounce\_halt\_rate}}
        \label {kw-bounce-halt-rate}
         
        \default {50}

	\texttt {NOT USED YET}

	This is the site default value for \lparam {bounce}.
	Above this rate of bouncing addresses in a list, messages
	for the list aren't distributed.

\subsection {\cfkeyword {welcome\_return\_path}}
        \label {kw-welcome-return-path}
         
        \default {owner}

	If set to string \texttt {unique}, sympa will use a unique email adresse in the
        return path starting with \texttt {bounce+} in order to remove the related subscriber.
        You need the \texttt {bounced} deamon and the plussed aliases as in 
        sendmail 8.7 and above.

\subsection {\cfkeyword {remind\_return\_path}}
        \label {kw-remind-return-path}
         
        \default {owner}

        Like \cfkeyword {welcome\_return\_path} but efficient with the remind message.

\section {Priority related}

\subsection {\cfkeyword {sympa\_priority}}  
        \label {kw-sympa-priority}

	\default {1}

        Priority applied to \Sympa commands while running the spool.

        Available since release 2.3.1.

\subsection {\cfkeyword {request\_priority}}  
        \label {kw-request-priority}

	\default {0}

        Priority for processing of messages for \samplelist-request,
	ie for owners of the list.

        Available since release 2.3.3

\subsection {\cfkeyword {owner\_priority}}  
        \label {kw-owner-priority}

	\default {9}

        Priority for processing messages for \samplelist-owner in
	the spool. This address will receive non-delivery reports
	(bounces) and should have a low priority.

        Available since release 2.3.3


\subsection {\cfkeyword {default\_list\_priority}}  
        \label {kw-default-list-priority}

	\default {5}

        Default priority for messages if not defined in list
        configuration file.

        Available since release 2.3.1.

\section {Database related}

The following parameters are needed when using a RDBMS and not
required otherwise:

\subsection {\cfkeyword {db\_type}}

        Database management system used (e.g. mysql, Pg, Oracle)
	
	This corresponds to the DataBase Driver (DBD) name and
	is therefore case-sensitive.

\subsection {\cfkeyword {db\_name}} 

	\default {sympa}

        Name of the database containing user information. See
        detailed description on database structure, \ref{rdbms-struct},
        page~\pageref{rdbms-struct}.

\subsection {\cfkeyword {db\_host}}

        Database host name.

\subsection {\cfkeyword {db\_user}}

        User with read access on the database.

\subsection {\cfkeyword {db\_passwd}}

        Password for \cfkeyword {db\_user}.

\section {Loop prevention}

   The following define your loop prevention policy for commands.
(see~\ref {loop-detection}, page~\pageref {loop-detection})

\subsection {\cfkeyword {loop\_command\_max}}

	\default {200}

	The maximum number of command reports sent to an email
	address. When it is reached, messages are kept as BAD
	and reports are not sent anymore.

\subsection {\cfkeyword {loop\_command\_sampling\_delay}} 

	\default {3600}

	This parameter defines the delay, expressed in seconds, over 
	which the counter of sent report (to an email address) is decreased.

\subsection {\cfkeyword {loop\_command\_decrease\_factor}} 

	\default {0.5}

	This is the decreasing factor (\texttt {from 0 to 1}), used to
	determine the new report counter when the delay is over.

\section {S/MIME configuration}

\Sympa optionaly can verify and use S/MIME signature for security purpose.
In this case, two optionel parameters must be assigned by the listmaster,
see \ref {smimeconf},  page~\pageref {smimeconf}.

\begin{itemize}

  \item \cfkeyword {openssl} : the path for the openssl binary file,
         
  \item \cfkeyword {trusted\_ca\_options} : the option used by openssl
        for trusted CA certificates. Required if cfkeyword {openssl} is defined.
	
  \item \cfkeyword {password} : the password for list's private key encryption. If not
	define \Sympa assume that list private key are not crypted.
\end{itemize}

%%%%%%%%%%%%%%%%%%%%%%%%%%%%%%%%%%%%%%%%%%%%%%%%%%%%%%%%%%%%%%%%%%%%%%%%%%%%%%
% WWSympa
%%%%%%%%%%%%%%%%%%%%%%%%%%%%%%%%%%%%%%%%%%%%%%%%%%%%%%%%%%%%%%%%%%%%%%%%%%%%%%

\cleardoublepage
\chapter {WWSympa}


WWSympa is \Sympa's web interface.

\section {Organization}
\label {wwsympa}

\WWSympa is fully integrated with \Sympa. It uses \file {sympa.conf}
and \Sympa's libraries. Default \Sympa installation will also
install WWSympa.

Every single piece of HTML in \WWSympa is out of the CGI code,
in template files (See \ref {tpl-format}, page~\pageref {tpl-format}).
This eases internationalization of pages, as well as per-site
customization. 

The code consists of one single Perl CGI script, \file {wwsympa.fcgi}.
To ensure good performences you can configure \WWSympa to use
FastCGI ; the CGI will be persistent in memory.\\
Every data will be accessed through the CGI, including web archives.
This is required to allow the authentication scheme to be applied
on everything.

Authentication is based on passwords stored in the database table
user\_table ; if the the appropriate \file {Crypt::CipherSaber} is
installed, password are encrypted in the database using reversible
encryption based on RC4. Otherwise they are stored in clear form.
In both case  reminding of passwords is possible.
 To keep track of authentication information, \WWSympa
uses HTTP cookies, stored on the client side. The HTTP cookie only 
indicates that a specified email address has been authentified ;
privileges are evaluated when an action is requested.

The web interface is unique for listmaster, list owners, subscribers,
others. They may access the same URLs with a different view of it, depending of their privileges. 

\WWSympa's main loop's algorithm is roughly : 
\begin {enumerate}
	\item Check of authentication informations returned by 
	the HTTP cookie

	\item Evaluation of the privileged of the user for the
        requested action 

	\item Processing of the requested action 

	\item Setup of variables resulting from the action 

	\item Parsing of the HTML template files
\end {enumerate}

\section {Installation}

%\begin {htmlonly}
%<A NAME='installwwsympa'>
%\end {htmlonly}


\begin{enumerate}

\item Edit \Sympa's main Makefile to customize destination dir, conf location,...

\item Install \Sympa

\item \%make installwws

\item edit your \file {/etc/wwsympa.conf} ; here are the recognized parameters :

 	\begin{itemize}

	\item default\_home \\
	\default {home} \\
        What action perform for WWSympa home page. If you have just a few list
	the default 'home' that present the list of lists organised by topics
        should be change for 'lists' a alphabetical list of lists.

	\item arc\_path \\
	\default {/home/httpd/html/arc} \\
	Where to store html archives. This parameter is used
        by \file {archived.pl} daemon. It is a good idea to install archive
        out of the web hierarchy to prevent a back door in the access control
        powered by WWSympa. Never mind you may have to install those archives
        under apache tree if it's configured with a chroot.

	\item archive\_default\_index thrd | mail \\
	\default {thrd} \\
	The default index when enterring web archives ; for either threaded or
	chronological order.

	\item archived\_pidfile  \\
	\default {archived.pid} \\
	The file containing the PID of \file {archived.pl}.

	\item bounce\_path \\
	\default {/var/bounce} \\
	Root directory for storing bounces (non-delivery reports). This parameter
	is mainly used by \file {bounced.pl} daemon.

	\item bounced\_pidfile  \\
	\default {bounced.pid} \\
	The file containing the PID of \file {bounced.pl}.

	\item cookie\_expire \\
	\default {30}
	HTTP cookies lifetime (in minutes). This is the default value
	used when users didn't set it.

	\item cookie\_domain \\
	\default {localhost} \\
	Domain for the HTTP cookies. If beginning with a dot ('.'),
	cookie is available within the specified internet domain,
	otherwise, for the specified host. Example : 
		\begin {quote}
		\begin{verbatim}
		   cookie_domain cru.fr
		   cookie is available for host 'cru.fr'

		   cookie_domain .cru.fr
		   cookie is available for any host within 'cru.fr' domain
		\end{verbatim}
		\end {quote}
	Default is nice unless \WWSympa's authentication process
        is shared with some other application on some others host.  

	\item icons\_url \\
	\default {/icons} \\
	URL of WWSympa's icons directory.

      	\item log\_facility \\

	WWSympa will log using this facility, default is \Sympa's syslog
        facility.
	Configure your syslog according to this parameter

	\item mhonarc \\
	\default {/usr/bin/mhonarc} \\
	Path to the wonderfull MHonArc program. Required for html archives
	http://www.oac.uci.edu/indiv/ehood/mhonarc.html

	\item title \\
	\default {Mailing Lists Service} \\
	The name of your mailing lists service. It will appear in
	the Title section of WWSympa.

      	\item use\_fast\_cgi   0 | 1 \\
	\default {1} \\
	Choose to use or not FastCGI. On listes.cru.fr using FastCGI 
	with WWSympa fasten up to 10 times. Refer to 
       	\htmladdnormallink {http://www.fastcgi.com/} {http://www.fastcgi.com/}
	and the apache config section of this document for details about 
	FastCGI.

	\end{itemize}

\item set \file {sympa.conf} parameter \cfkeyword {wwsympa\_url}
	to the base URL of your \WWSympa

\end{enumerate} 

\section {MhOnArc}
 
MHonArc is a nice converter from mime messages to html. Refer to
\htmladdnormallink {http://www.oac.uci.edu/indiv/ehood/mhonarc.html}
{http://www.oac.uci.edu/indiv/ehood/mhonarc.html}.

The long mhonarc ressource file is used in a special way with \WWSympa
because mhonarc is used to produce not html document but only part of them
that are included in a complete document starting with <HTML> and terminating
with </HTML> ;-)
The best way to start is to use the MhOnArc ressource file as
provided in \WWSympa distribution. 

The mhonarc ressources  file name is \file {mhonarc-ressources}. 
You may locate this file either in \begin{enumerate}
 	\item \tildedir {sympa/expl/\samplelist/mhonarc-ressources}
	in order to create specific archive look for one list

	\item or \tildedir {sympa/bin/mhonarc-ressources}

\end{enumerate}

\section {Archiving daemon}

\file {archived.pl} is intended to convert messages coming from \Sympa's spools 
and to call \file {mhonarc} to create html version located according to the 
"arc\_path" wwsympa parameter. You should probably install these archives 
out of \Sympa home\_dir. (we think the initial choice of \Sympa to 
store mail archive in the \tildedir {sympa/expl/\samplelist} directory). Note that html archive 
contains a text version of any message and are totaly separed from \Sympa's main archive.

\begin{enumerate}

\item create a directory according "arc\_path" wwsympa parameter. (must be owned by sympa)
     do not need to be under apache space unless your server runs a chroot.

\item foreach list, if you need web archive create a new webarchive pragraph in list config
     example :
\begin {quote}
\begin{verbatim}
     web_archive
     access public|private|owner|listmaster|closed
\end{verbatim}
\end {quote}

     If web\_archive is define for a list any message distributed by this list is copied
     in \tildedir {sympa/spool/outgoing/}. (No need to create crazy subscribers that receive
     a copy of messages)

\item launch \file {archived.pl}. It MUST run under the same uid/gid then 
\Sympa and Apache
 
\item check \WWSympa logs or start \file {archived.pl} in debug mode (-d). 

\item If you change mhonarc ressources, in order to rebuild all the archive 
using the new look define for mhonarc, just create a empty file owned by 
\Sympa  in \tildedir {sympa/spool/outgoing} named ".rebuild.\samplelist@myhost". 

\begin {quote}
\begin{verbatim}
     example : su sympa -c "touch ~sympa/spool/outgoing/.rebuild.sympa-fr@cru.fr"
\end{verbatim}
\end {quote}
You can also rebuild web archives from within the admin page of the list.

\end{enumerate}
 
\section {HTTPD setup}

\subsection {wwsympa.fcgi access privilege}
 
      
     Because Sympa and WWsympa share a lot of files \file {wwsympa.fcgi},
     must run under same 
     uid/gid than \file {archived.pl}, \file {bounced.pl} and \file {sympa.pl}.
     This can be done by various configuration :
\begin{itemize}
\item On various operating system, you have nothing to set up because
      wwsympa.fcgi is installed with suid and sgid bits, but this may failed
      if script using Suid are refused by your system.

\item Run a dedicated Apache server under sympa.sympa uid/gid (Apache default
      is nobody.nobody)

\item Use a virtual Apache server with sympa.sympa uid/gid ; you need a apache
      compiled with suexec.

\item otherwise, you can skip operating protection about Suid scripts
      using a short C program who start \file {wwsympa.fcgi} and who is
      owned by sympa and as the Suid bit. Here is a example (without garanty).
\begin {quote}
\begin{verbatim}

#include <unistd.h>

#define WWSYMPA "/home/sympa/bin/wwsympa.fcgi"

int main(int argn, char **argv, char **envp) {
  execve(WWSYMPA,argv,envp);
}

\end{verbatim}
\end{quote}
\end{itemize}

\subsection {Instaling wwsympa.fcgi in your Apache server}
     If you chose to run \file {wwsympa.fcgi} as a simple CGI, you just need to
     script alias it. 

\begin {quote}
\begin{verbatim}
     Example :
       	ScriptAlias /wws /home/sympa/bin/wwsympa.fcgi
\end{verbatim}
\end{quote}

     Running  FastCGI will provide much faster responses from your server and 
     lower load (to understand why, read 
     \htmladdnormallink 
     {http://www.fastcgi.com/fcgi-devkit-2.1/doc/fcgi-perf.htm}
     {http://www.fastcgi.com/fcgi-devkit-2.1/doc/fcgi-perf.htm})
     
\begin {quote}
\begin{verbatim}
     Example :
	FastCgiServer /home/sympa/bin/wwsympa.fcgi -processes 2
	<Location /wws>
   	  SetHandler fastcgi-script
	</Location>

	ScriptAlias /wws /home/sympa/bin/wwsympa.fcgi

 \end{verbatim}
\end{quote}
    

\section {Database configuration}

\WWSympa needs a RDBMS (Relational Database Management System) to
run. All database access is performed via \Sympa API. \Sympa
currently interfaces with \htmladdnormallink {MySQL}
{http://www.mysql.net/}, \htmladdnormallink {PostgreSQL}
{http://www.postgresql.pyrenet.fr/}, \htmladdnormallink {Oracle}
{http://www.oracle.com/database/} and \htmladdnormallink {Sybase}
{http://www.sybase.com/index_sybase.html}.

Database is needed to store user passwords and preferences.
The database structure is documented in \Sympa documentation ;
scripts for creating it are also provided with \Sympa distribution
(in \dir {script}). 

User information (password and prefs) are stored in �User� table.
User passwords stored in database are encrypted using reversible
RC4 encryption controled with the \cfkeyword {cookie} parameter
since \WWSympa
might need to remind users with their passwords. 
\WWSympa security relies on your database's security. 


\section {Sharing \WWSympa authentication with other applications}

You might want to make other web applications collaborate with \Sympa.
Then you need to share the same authentication system. \Sympa uses
HTTP cookies to carry user's auth information from page to page.
This cookie carries no priviliege information. To make your application
work with \Sympa you have two solutions :

\begin {itemize}

\item Delegating authentication operations to \WWSympa \\
If you want to save time programming a CGI that does Login, Logout
and Remindpassword, you can use \WWSympa's login page from your 
application and then exploits the cookie information in your application. 
The cookie format is :
\begin{verbatim}
user=<user_email>:<md5>
\end{verbatim}

where <user\_email> is the user's complete email address and
<md5> if a MD5 checksum of the <user\_email>+\Sympa \cfkeyword {cookie}
configuration parameter.
Your application needs to know what the \cfkeyword {cookie} parameter
is, so it can check the HTTP cookie validity ; this is a shared secret
between \WWSympa and your application.
\WWSympa's \textit {loginrequest} page can be called to return to the
referer URL when action is performed. Here is a sample HTML anchor :

\begin{verbatim}
<A HREF="/wws/loginrequest/referer">Login page</A>
\end{verbatim}


\item Using \WWSympa's HTTP cookie format within your auth module \\
To cooperate with \WWSympa, all you need to do is adopt its HTTP
cookie format and share its secret, used to generate MD5 checksum,
ie the \cfkeyword {cookie} configuration parameter. Thereby \WWSympa
will accept users authentified through your application without
further authentication.


\end {itemize}

%%%%%%%%%%%%%%%%%%%%%%%%%%%%%%%%%%%%%%%%%%%%%%%%%%%%%%%%%%%%%%%%%%%%%%%%%%%%%%
% Using Sympa with a RDBMS
%%%%%%%%%%%%%%%%%%%%%%%%%%%%%%%%%%%%%%%%%%%%%%%%%%%%%%%%%%%%%%%%%%%%%%%%%%%%%%

\cleardoublepage
\chapter {Using \Sympa with a RDBMS}
\label {sec-rdbms}

It is possible for \Sympa to store its user information within
a Relational DataBase System. Currently you can use one of the following
DBMS : MySQL, PostgreSQL, Oracle, Sybase. Interfacing with other RDBMS
requires only a few changes in the code since the API used, 
\htmladdnormallinkfoot {DBI} {http://www.symbolstone.org/technology/perl/DBI/} 
(DataBase Interface), has DBD (DataBase Drivers) for many DBMS.

\section {Prerequisites}

You need to have a DataBase System installed (not necessarily 
on the same host as \Sympa) and the client librairies for that
Database installed on the machine, as far as a Perl DBD (DataBase
Driver) is provided for it. 
Check the \htmladdnormallinkfoot
{DBI Module Availability} {http://www.symbolstone.org/technology/perl/DBI/}.

\section {Installing Perl modules}

\Sympa will use DBI to communicate with the database system and
therefore requires the DBD for your database system. DBI and 
DBD::YourDB (Msql-Mysql-modules for MySQL) are distributed as 
CPAN modules. Refer to ~\ref 
{Install other required CPAN modules}, page~\pageref 
{Install other required CPAN modules} for installation
details of these modules.

\section {Creating sympa DataBase}

\subsection {Database structure}

The sympa database structure is slightly different from the
structure of a \file {subscribers} file. A \file {subscribers}
file is a text file based on paragraphs (almost similar to 
the \file {config} file) ; each paragraph completely describes 
a subscriber. If somebody is subscribed to two lists, he/she 
appears in both subscribers files.

The DataBase distinguishes information relative to a person (email,
real name, password) and his/her subscription options (list
concerned, date of subscription, reception option, visibility 
option). This results in separating the data in two tables :
user\_table and subscriber\_table joined together by a user/subscriber email.

\subsection {Database creation}

\file {create\_db} script below will create the sympa database for 
you. You can find it in the \dir {script/} directory of the 
distribution (currently scripts are available for mysql, postgresql, Oracle and Sybase).

\begin {quote}
\begin{verbatim}

## MySQL Database creation script
CREATE DATABASE sympa;

## Connect to DB 
\r sympa

CREATE TABLE user_table (
  	email_user          	varchar (100) NOT NULL,
  	gecos_user          	varchar (150),
  	password_user		varchar (40),
	cookie_delay_user	int,
	lang_user		varchar (10),
	PRIMARY KEY (email_user)
);

CREATE TABLE subscriber_table (
  	list_subscriber       	varchar (50) NOT NULL,
	user_subscriber		varchar (100) NOT NULL,
	date_subscriber		datetime NOT NULL,
	visibility_subscriber	varchar (20),
	reception_subscriber	varchar (20),
	bounce_subscriber	varchar (30),
	update_subscriber 	datetime,
	PRIMARY KEY (list_subscriber, user_subscriber)
);

\end{verbatim}
\end {quote}

You can execute the script using a simple SQL shell such as
mysql or psql.

Example:

\begin {quote}
\begin{verbatim}
# mysql  < create_db.mysql
\end{verbatim}  
\end {quote}

\subsection {Importing data from subscribers files}

You might need to import subscribers data from an older
version of \Sympa into sympa database. Simply edit the 
list configuration using \WWSympa (this requires listmaster privileges) 
and change the data source from \textbf {file} to \textbf {database}. 



\section {\Sympa configuration}

To store subscribers information in your newly created
database, you first need to tell \Sympa what kind of
database he's going to work with, then you must configure
your list to work with the database.

You define the database source in \file {sympa.conf} :
\cfkeyword {db\_type}, \cfkeyword {db\_name}, 
\cfkeyword {db\_host}, \cfkeyword {db\_user}, 
\cfkeyword {db\_passwd}.

If you are interfacing \Sympa with an Oracle database, 
\cfkeyword {db\_name} is the SID.

All your lists are now configured to use the database,
unless you set list parameter \lparam {user\_data\_source} 
to \textbf {file} or \textbf {include}. 

\Sympa will now extract and store user
information for this list from the database instead of using
\file {subscribers} file. Though subscribers information are 
dumped in \file {subscribers.db.dump} at every shutdown 
to allow hand rescue restart (by renaming subscribers.db.dump to
subscribers and changing user\_data\_source parameter) if the database 
is out of order.

%%%%%%%%%%%%%%%%%%%%%%%%%%%%%%%%%%%%%%%%%%%%%%%%%%%%%%%%%%%%%%%%%%%%%%%%%%%%%%
% SMIME
%%%%%%%%%%%%%%%%%%%%%%%%%%%%%%%%%%%%%%%%%%%%%%%%%%%%%%%%%%%%%%%%%%%%%%%%%%%%%%


\cleardoublepage
\chapter {\Sympa with S/MIME and HTTPS}
    \label {smime}

S/MIME is a cryptographic method for Mime messages based on X509 certificate.
Before to install \Sympa S/Mime features named S/Sympa, you must be
persuaded that  ``S/MIME'' mean ``Secure MIME'' and has nothing to
do with ``Simple''.

This chapter just describes what security level is provided by \Sympa while
using S/MIME messages and how to configure \Sympa for it. It is not intended
to teach anyone what S/Mime is and why it is so complex ! RFCs number 2311,
2312, 2632, 2633 and 2634  and a lot of literature about S/MIME, PKCS\#7
and PKI is available on the Internet. \Sympa 2.7 is the first version of
\Sympa including S/MIME features, this is beta-testing features.

\section {Signed messages distribution}

Nothing to do.
You probably imagine that any mailing list manager (or any mail forwarder)
is compatible with S/MIME signature as long as it respects MIME structure of incomming
messages. You are right. Even Majordomo can distribute a signed message !
As \Sympa provides nice MIME compatibility, You don't need to do
anything in order to allow subscribers to verify signed messages distributed
through a list. This is not a performence at all, any processus that
distributes messages  are compatible with end user
signing processus. Sympa just skips message footer attachement (ref \ref {messagefooter}, page~\pageref {messagefooter}) to prevent any
body corruption which would break the signature.

\section {Use of S/MIME signature by Sympa itself}

Sympa is able to verify S/MIME signatures in order to apply S/MIME
authentication method for the message handling. 
Currently, this feature is limited to the
distribution process and to commands when sympa find them in the message
body.  The reasons of this limitation is related to current S/MIME
usage.
S/MIME signature structure is based on a encryption of a digest of the
message. Most S/MIME agent does not include any part of the
message headers in the message digest so anyone can modify the message
header without signature corruption ! This is easy to do, for example, anyone
can edit a signed message with his prefered message agent, modify whatever
header he want (for example the \texttt {Subject:} , the \texttt {Date:} and
the \texttt {To:} and redistribute the message in a list or to the robot
without breaking the signature.

So we can not apply S/MIME
authentication method to command parsed in the \texttt {Subject:} field of a
message or via the \texttt {-subscribe} or \texttt {-unsubscribe} email
adresse. 

\section {Use of S/MIME encryption} 

S/Sympa is not a implementation of ``S/MIME Symmetric Key Distribution''
internet draft. This sophisticated scheme is requiered for large lists
with encryption. So, there is still some task for later developpement :) 

We assume that S/Sympa distribute message as received, ie uncrypted if the
list received a uncrypted message, crypted otherwise.

In order to be able to send encrypted message to a list, the sender need
to use the X509 certificat of the list. Sympa will send a crypted message
to each subscriber using subscribers certificate. To provide this feature,
\Sympa needs to manage one certificate for each list and one for each
subscriber. This is availible in Sympa version number 2.8 and above.

\section {S/Sympa configuration} 

\subsection {Installation}
\label {smimeinstall}

All you need is to install openssl (http://www.openssl.org), the minimum requirement is
the \texttt {OpenSSL 0.9.5a}. OpenSSL is used by \Sympa as an external plugin
(like sendmail or postfix), so it must be installed with the apropriate access
(x for sympa.sympa). 

\subsection {configuration in sympa.conf}
\label {smimeconf}

S/Sympa configuration is very simple. If you are used to Apache SSL, you
will not be desorientate. If you are OpenSSL guru, you will
feel at home and perhaps you will suggest us some changes.
 
The minimum is to teach Sympa where is the binary file for openssl program
and where are the certificates of the trusted certificate authority. 
This is done using the optional parameters \unixcmd {openssl} and
\cfkeyword {trusted\_ca\_options}.
\begin{itemize}

  \item \cfkeyword {openssl} : the path for the openssl binary file,
         unsually \texttt {/usr/local/ssl/bin/openssl}
  \item \cfkeyword {trusted\_ca\_options} : the option used by openssl
        for trusted CA certificates. 
        The file \cfkeyword {/home/sympa/bin/etc/ca-bundle.crt} is distributed
        with Sympa and describes a wellknown set of CA as the default Netscape
        navigator configuration. You can use this set of certificate as trusted
        setting  \texttt {trusted\_ca\_options -CAfile /home/sympa/bin/etc/ca-bundle.crt}.
        You can also used the \cfkeyword {-CApath} \unixcmd {openssl} option or both 
        \cfkeyword {-CApath} and \cfkeyword {-CAfile}. Example :       
        \texttt {trusted\_ca\_options -CApath /home/sympa/etc/ssl.crt -CAfile /home/sympa/bin/etc/ca-bundle.crt}.

	Both  \cfkeyword {-CAfile} file and \cfkeyword {-CApath} directory
        should be share with your Apache+mod\_ssl configuration. It is usefull
	for S/Sympa web interface.  Please refer to Openssl documentation for details.
  \item \cfkeyword {key\_passord} : the password use to protect all list private 	key. xxxxxxx	
\end{itemize}


\subsection {configuration to recognize S/MIME signature}
\label {smimeforsign}

Once  \texttt {openssl} installed and  \texttt {sympa.conf} configured,
you S/Sympa is ready to use S/Mime signature for any operation authentication. You just need to use the appropriate scenario for the operation you want
to secure. 
(see \ref {scenarii}, page~\pageref {scenarii}).

When receiving a message, \Sympa apply
the scenario  with the apropriate authentication method parameter.
In most case, the authentication method is ``\texttt {smtp}'', in case
the message is signed and the signature as been checked and match the
sender email, \Sympa apply ``\texttt {smime}'' authentication
method.

Take care that if the scenario ignores this authentication method, the
operation requested will be rejected. Subsequently, scenario from version
previous to 2.7 are not compatible with the OpenSSL configuration of Sympa. 
All
standard scenario (thoses who are distributed with sympa)
now includes \texttt {smime} method. The following example is
named \texttt {send.private\_smime} limit send to subscribers using S/mime signature :

\begin {quote}
\begin{verbatim}
title.us restricted to subscribers check smime signature
title.fr limit� aux abonn�s, v�rif de la signature smime

is_subscriber([listname],[sender])             smime  -> do_it
is_editor([listname],[sender])                 smime  -> do_it
is_owner([listname],[sender])                  smime  -> do_it
\end{verbatim}
\end {quote}

It as also possible to mixe various authentication method in a single scenario ; this one
\texttt {send.private\_key} require either md5 return key or S/Mime signature :
\begin {quote}
\begin{verbatim}
title.us restricted to subscribers with previous md5 authentication
title.fr r�serv� aux abonn�s avec authentification MD5 pr�alable

is_subscriber([listname],[sender]) smtp          -> request_auth
true()                             md5,smime     -> do_it
\end{verbatim}
\end {quote}

\subsection {distributing encrypted messages}
\label {smimeforencrypt}

In this section describe S/Sympa encryption features. The goal is to use
S/MIME encryption list subscribers distribution if the message as been
received encrypted from its sender. 

Why S/Sympa is concerned by S/MIME encryption distribution process ?
That's because encryption is made using the \textbf {recepient} X509
certificat whereas signature need the sender private key. So a encrypted
message can be read by the recepient only if he is owner of the private
key associated with the certificat.
Therefore the only way to encrypt a message for a list of recepients is
to encrypt and send the message foreach recepient. That's what S/Sympa
do when distributing a encrypted message.

The S/Sympa encryption feature in the distribution process suppose that sympa
as received a encrypted message for some list. To be able to encrypt a message
for a list, the sender must have some access to a X509 certificat of the list.
So the first think to do is to install a certificat and a private key for
the list.
The processus to manage and to obtain certificats is complex. Current version
of S/Sympa assume that list certificat and private key are installed by
the listmaster.
It is a nice idea to have a look at OpenCA (http://www.openca.org)
documentation and/or PKI providers web documentation.
You can use commercial certificats or self made ones. Of course, the
certificat must ,be approved for email aplication and issued by one of
the trusted CA described in \cfkeyword {-CAfile} file or
\cfkeyword {-CApath} OpenSSL option. 


The list private key must be installed in a file named
\tildedir {sympa/expl/\samplelist/private\_key}. All the list private
keys must be encrypted using a single password defined by the
parameter \cfkeyword {password} in \cfkeyword {sympa.conf}.


\subsubsection {Use of Netscape navigator to get X509 list certificat}

In many case e-mail X509 certificate are distributed vis a web server and
loaded in the browser using your mouse :) Netscape allow to exports
certificate in a file. So one way to get list certificat is to get e-mail
certificate for the list canonical adress in your brower and to export and install it for Sympa :
\begin {enumerate}
\item browse the net and load a certificate for the list adress on some
PKI provider (your own OpenCa pki server , thawte, verisign, ...). Be
carefull :  the certificat email must be exctly the canonical adress of
your  list, otherwise, the signature will be incorrect (sender email will
not match signer email).
\item on the security menu, select the intended certificate and export
it. Netscape will promt you for a password and a filename to encrypt
the output file. The format use by Netscape is  ``pkcs\#12''. 
Copy this file in the list home directory.
\item convert the pkcs\#12 file into a pair of pem file :
\cfkeyword {cert.pem} and \cfkeyword {private\_key} using
the \unixcmd {~sympa/bin/p12topem.pl} script. Use \unixcmd
{p12topem.pl -help} for details.
\item be sure that \cfkeyword {cert.pem} and \cfkeyword {private\_key}
are owned by sympa with ``r'' access.
\item As soon as certificat is installed for a list the list  home page
include a new link to load the certificat in users brower, and the welcome
messages is signed by the list.
\end {enumerate} 

%%%%%%%%%%%%%%%%%%%%%%%%%%%%%%%%%%%%%%%%%%%%%%%%%%%%%%%%%%%%%%%%%%%%%%%%%%%%%%
% Customization
%%%%%%%%%%%%%%%%%%%%%%%%%%%%%%%%%%%%%%%%%%%%%%%%%%%%%%%%%%%%%%%%%%%%%%%%%%%%%%

\cleardoublepage
\chapter {Customizing \Sympa/\WWSympa}
    \label {customization}

\section {Template files format}
\label{tpl-format}
\index{templates format}

Template files within \Sympa and \WWSympa are text files containing 
programming elements (variables, conditions, loops, file inclusions)
that will be parsed in order to adapt to the runtime context. These 
templates are an extension of programs and therefore give access to 
a limited list a variables (those defined in the '\textit {hash}' 
parameter given to the parser). 

Review the Site template files (\ref {site-tpl}, page~\pageref {site-tpl}) 
and List template files (\ref {list-tpl}, page~\pageref {list-tpl}).

Let us describe the syntaxic elements of templates.

\subsection {Variables}

Variables are enclosed between brackets '\textit {[]}'. The variable name
is composed of alphanumericals (0-1a-zA-Z) or underscores (\_).
The syntax for accessing an element in a '\textit{hash}' is [hash-$>$elt].

\textit {Examples:} 
\begin {quote}
\begin{verbatim}
[url]
[is_owner]
[list->name]
[user->lang]
\end{verbatim}
\end {quote}

Foreach template you wish to customize, check the available variables in the
documentation.

\subsection {Conditions}

Conditions include variables comparisons (= and <>) or existance.
Conditions syntaxic elements are [IF xxx], [ELSE], [ELSIF xxx] and
[ENDIF].

\textit {Examples:} 
\begin {quote}
\begin{verbatim}
[IF  user->lang=fr]
Bienvenue dans la liste [list->name]
[ELSIF user->lang=es]
Bienvenida en la lista [list->name]
[ELSE]
Welcome in list [list->name]
[ENDIF]

[IF is_owner]
The following commands are available only 
for lists's owners or moderators:
....
[ENDIF]
\end{verbatim}
\end {quote}

\subsection {Loops}

Loops allows to go through a list of elements (internally represented by a 
'\textit{hash}' or an '\textit{array}'). 

\texttt{Example :}
\begin {quote}
\begin{verbatim}
A review of public lists

[FOREACH l IN lists]
   [l->NAME] 
   [l->subject]
[END]
\end{verbatim}
\end {quote}

\texttt {[elt-$>$NAME]} is a special element of the current entry providing 
the key in the '\textit{hash}' (here the list name). When going the
an '\textit{array}', \texttt{[elt-$>$INDEX]} is the index of the current
entry.

\subsection {File inclusions}

You can include another file within a template . The specified file can be 
included asis or parsed itself (there is no loop detection). The file 
path is either specified in the directive or accessed in a variable.

Inclusion of a text file :

\begin {quote}
\begin{verbatim}
[INCLUDE 'archives/last_message']
[INCLUDE file_path]
\end{verbatim}
\end {quote}

The first example includes a file which relative path is \file {archives/last\_message}.
The second example includes a file which path is in file\_path variable.

Inclusion and parsing of a template file :

\begin {quote}
\begin{verbatim}
[PARSE 'welcome.tpl']
[PARSE file_path]
\end{verbatim}
\end {quote}

The first example includes the template file \file {welcome.tpl}.
The second example includes a template file which path is in file\_path variable.

\subsection {Stop parsing}

You may need to exclude lines in a template out of the parsing
process. You can perform this by stopping and restarting the
parsing.

Escaping sensible Javascript functions :

\begin {quote}
\begin{verbatim}
<HEAD>
<SCRIPT LANGUAGE="JavaScript">
<!-- for other browsers
  function toggle_selection(myfield) {
    for (i = 0; i < myfield.length; i++) {
    [STOPPARSE]
       if (myfield[i].checked) {
            myfield[i].checked = false;
       }else {
	    myfield[i].checked = true;
       }
    [STARTPARSE]
    }
  }
// end browsers -->
</SCRIPT>
</HEAD>
\end{verbatim}
\end {quote}


\section {Site template files}
\label{site-tpl}
\index{templates, site}

These files are used by Sympa as service messages for \mailcmd {HELP}, 
\mailcmd {LISTS} and \mailcmd {REMIND *} commands. These files are interpreted 
(parsed) by \Sympa and respect the template format ; every file has a .tpl extension. 
See \ref {tpl-format}, 
page~\pageref {tpl-format}. 

Sympa looks for these files in the following order (where $<$list$>$ is the
listname if defined, $<$action$>$ is the name of the command and $<$lang$>$ is
the prefered language of the user) :
\begin {enumerate}
	\item \tildedir {sympa/expl/$<$list$>$/$<$action$>$.$<$lang$>$.tpl}. 
	\item \tildedir {sympa/expl/$<$list$>$/$<$action$>$.tpl}. 
	\item \tildedir {sympa/etc/templates/$<$action$>$.$<$lang$>$.tpl}. 
	\item \tildedir {sympa/etc/templates/$<$action$>$.tpl}. 
	\item \tildedir {sympa/bin/etc/templates/$<$action$>$.$<$lang$>$.tpl}.
	\item \tildedir {sympa/bin/etc/templates/$<$action$>$.tpl}.
\end {enumerate}

If the file starts with a From: line it is considered as
a full message and will be sent (after parsing) without adding SMTP
headers. Else file is taken as a text/plain message body.

The following variables may be used in these template files :

\begin {itemize}

	\item[-] [conf-$>$email] : sympa email address local part

	\item[-] [conf-$>$host] : sympa host name

	\item[-] [conf-$>$sympa] : sympa's complete email address

	\item[-] [conf-$>$wwsympa\_url] : \WWSympa root URL

	\item[-] [conf-$>$listmaster] : listmaster email addresses

	\item[-] [user-$>$email] : user email address

	\item[-] [user-$>$gecos] : user gecos field (usually his/her name)

	\item[-] [user-$>$password] : user password

	\item[-] [user-$>$lang] : user language	

\end {itemize}

\subsection {helpfile.tpl} 


	This file is sent in response to a \mailcmd {HELP} command. 
	You may use additional variables
\begin {itemize}

	\item[-] [is\_owner] : TRUE if the user is list owner

	\item[-] [is\_editor] : TRUE if the user is list editor

\end {itemize}

\subsection {lists.tpl} 

	File returned by \mailcmd {LISTS} command. 
	An additional variable is available :
\begin {itemize}

	\item[-] [lists] : this is a hashtable indexed by list names
			containing lists' subjects. Only lists visible
			for this user (refering to \lparam {visibility} 
			list parameter avec listed.
\end {itemize}

\textit {Example:} 

\begin {quote}
\begin{verbatim}
Here are public lists for [conf->email]@[conf->host]

[FOREACH l IN lists]
	
 [l->NAME]: [l->subject]

[END]

\end{verbatim}
\end {quote}

\subsection {global\_remind.tpl} 

	This file is sent in used by \mailcmd {REMIND *} command. 
	(see~\ref {cmd-remind}, page~\pageref {cmd-remind})
	You may use additional variables
\begin {itemize}

	\item[-] [lists] : this is a array containig listnames the user
			is subscribed to.
\end {itemize}

\textit {Example:} 

\begin {quote}
\begin{verbatim}

This is a subscription reminder.

You are subscribed to the following lists :
[FOREACH l IN lists
	
 [l] : [conf->wwsympa\_url]/info/[l]

[END]

Your subscriber email : [user->email]
Your password : [user->password]

\end{verbatim}
\end {quote}

\section {Web template files}
\label{web-tpl}
\index{templates, web}

You may define your own web template files, different from the standard
ones. \WWSympa first looks for list specific web templates, then for
site web templates, at last uses its default. 

Your list web template files should be placed in \tildedir {sympa/expl/\samplelist/wws\_templates} 
directory ; your site web templates in \tildedir {sympa/expl/wws\_templates} directory.

\section {Internationalization}
\label {internationalization}
\index{internationalization}
\index{localization}

\Sympa has originally been designed as a multi-lingual Mailing List
Manager. First versions of \Sympa already separated messages from
the code itself, messages being stored in NLS catalogues (according 
to XPG4 standard). Later a \lparam{lang} list parameter was introduced.
Now \Sympa also keeps track of users' language preference.


\subsection {\Sympa internationalization}

Every message sent by \Sympa to users, owners, editors is outside
the code, in a message catalog. These catalogs are located in 
\tildedir{sympa/nls/} directory. These messages have currently been
translated into 6 different languages : 

\begin{itemize}

\item us: US english

\item fr: French

\item es: Spanish

\item de: German

\item it: Italian

\item cn-big5: BIG5 Chinese (Honk Kong, Taiwan)

\item cn-gb: GB Chinese (Mainland China)

\end{itemize}

To tell \Sympa to use a message catalog, you can either set 
the \cfkeyword{lang} parameter in \file{sympa.conf} or
set \file{sympa.pl} \texttt{-l} option on the command line.

\subsection {List internationalization}

\lparam{lang} list parameter defines the language for a list.
It is currently used by \WWSympa and to initialize user's
language preference at subscription time.

In future versions, all messages returned by \Sympa concerning
a list should be in the list's language. 

\subsection {User internationalization}

The user language preference is currently exploited by \WWSympa
only. There is no email-based command for a user to set his/her
language. The language preference is initialized when the user
subscribes to his/her first list. \WWSympa allows the user to change 
it.

\section {Topics}
\label{topics}
\index{topics}

\WWSympa's homepage shows a list of topics for classifying
mailing lists. This is dynamically generated, based on lists'
configuration parameter \lparam {topics}. A list may appear 
in multiple categories.

The list of topics is defined in \file {topics.conf} configuration
file in \tildedir {sympa/etc} directory. The format of this file is 
as follows :
\begin{quote}
\begin{verbatim}
<topic1_name>
title	<topic1 title>
visibility <topic1 visibility>
....
<topicn_name/subtopic_name>
title	<topicn title>
\end{verbatim}
\end{quote}

As you can see, you can use subtopics, the separator being \textit {/}.
The topic name is composed of alphanumericals (0-1a-zA-Z) or underscores (\_).
The order the topics are listed is respected in \WWSympa's homepage.
The \textbf {visibility} line defines who can view the topic (now available for subtopics).
It refers to the associated topics\_visibility scenario.
You will find a sample of \file {topics.conf} in \dir {sample} 
directory ; NONE is installed as default. 


\section {Scenarii}
    \label {scenarii}
    \index{scenario}

List parameters controlling commands' behaviour indicates a scenario.
Example : the parameter send private is related to the scenario send.private.
There are tree locations for scenario. When \Sympa applies a scenario, it looks first for it
into the related list directory \tildedir {sympa/expl/$<$list$>$/scenari}, if it is not found, it scans \tildedir {sympa/etc/scenari}
and then in \tildedir {sympa/bin/etc/scenari} which is the directory intall by the Makefile.

A scenario is a small configuration langage to describe who
can perform an operation and what authentication method is requested for it.
A scenario is a ordered set of rules. The goal is to provide a simple and
flexible way to configure authorization and authentication for each operation.


Each scenario rule contains :
\begin{itemize}
\item a condition : the condition is evaluated by \Sympa. It can use some
  variables such as $[$sender$]$ for the sender email $[$list$]$ for the listname etc.
\item a authentication method. The authentication method can be \cfkeyword {smtp},
\cfkeyword {md5} or \cfkeyword {smime}. The rule is applyed by \Sympa if both condition
and authentication method match the runtime context. \cfkeyword {smtp} is used if
\Sympa use the SMTP \cfkeyword {from:} header , \cfkeyword {md5} is used is a unique
md5 key as been returned by the requestor to valid is message, \cfkeyword {smime}
is used for signed messages (see \ref {smimeforsign}, page~\pageref {smimeforsign}).
\item a returned atomic action that will be executed by \Sympa if the rule match

\end{itemize}

 
Example

\begin{quote}
del.auth
\begin{verbatim}
title.us deletion performed only by list owners, need authentication
title.fr suppression r�serv�e au propri�taire avec authentification
title.es eliminaci�n reservada s�lo para el propietario, necesita autentificaci�n


  is_owner([listname],[sender])  smtp       -> request_auth
  is_listmaster([sender])        smtp       -> request_auth
  true()                         md5,smime  -> do_it
\end{verbatim}
\end{quote}

Scenari can also contain includes :

\begin{quote}
\begin{verbatim}
    subscribe
        include commonreject
        match([sender], /cru\.fr$/)          smtp,smime -> do_it
	true()                               smtp,smime -> owner
\end{verbatim}
\end{quote}
	    

In this case sympa apply reccursively the scenario named \texttt {include.commonreject}
before introducing the other rules. This as been introduced in
order to made a liter administration of common rules.

A bunch of scenari is provided with \Sympa distribution ; it provides
all possible configurations as defined in previous releases of \Sympa
($<$= 2.3) whithout any change in your list configuration files. 

These standard scenari are located in \tildedir {sympa/bin/scenari/}
directory. Default scenari are named <command>.default.

You may also define and name your own scenari. Store them in 
\tildedir {sympa/etc/scenari} directory. 
Example:

Copy the previous scenario in \file {scenari/subscribe.rennes1} :

\begin {quote}
\begin{verbatim}
equal([sender], 'userxxx@univ-rennes1.fr') smtp,smime -> reject
match([sender], /univ-rennes1\.fr$/) smtp,smime -> do_it
true()                               smtp,smime -> owner
\end{verbatim}
\end{quote}

You may now refer to this scenario in any list configuration file as :

\begin {quote}
\begin{verbatim}
subscribe rennes1
\end{verbatim}
\end{quote}

A scenario consists in rules, evaluated from the first to the last. 
Rules are defined as follows :
\begin {quote}
\begin{verbatim}
<rule> ::= <condition> <auth\_list> -> <action>

<condition> ::= [!] <condition
		| true ()
                | equal (<var>, <var>)
                | match (<var>, /perl_regexp/)
                | is_subscriber (<listname>, <var>)
                | is_owner (<listname>, <var>)
                | is_editor (<listname>, <var>)
                | is_listmaster (<var>)
<var> ::= [email] | [sender] | [list-><list\_key\_word>] | [conf-><conf\_key\_word>] | [header-><smtp\_key\_word>] |  <string>

<listname> ::= [listname] | <listname_string>

<auth\_list> ::= <auth>,<auth\_list> | <auth>

<auth> ::= smtp|md5|smime

<action> ::=   do_it [,notify]
             | do_it [,quiet]
             | reject
             | request_auth
             | owner

<list\_key\_word> ::= name | host | lang | max\_size | priority | reply\_to | 
		      status | subject | account | 

<conf\_key\_word> ::= host | email | listmaster | default\_list\_priority | 
		      sympa\_priority | request\_priority | lang | max\_size
	 	      
\end{verbatim}
\end{quote}

perl\_regexp can containt the string [host] (it will be interpreted at run time as the list or robot domain.
The variable notation [header-$>$<smtp\_key\_word>] is interpreted as the SMTP header value only when performing
the sending message scenario. It can be use for exemple to ask editor validation for multipart messages.

%[idees de scenario]

\section {Loop detection}
    \label {loop-detection}
    \index{loop-detection}

\Sympa uses multiple tools to avoid loops in Mailing lists

First it rejects messages that either come from a robot (based on the
From: and other header fields) or messages that contain commands.

Every message sent by \Sympa includes a X-Loop header field set to
the listname. If the message comes back, \Sympa will detect that
it has already been sent (unless X-Loop header fields have been
erased).

\Sympa keeps track of Message IDs and will refuse to send multiple
message with the same message ID in the same mailing list.

At last \Sympa can detect loops due to command reports. 
Example of loop:

\begin {quote}
\begin{verbatim}
1 - X sends a command to Sympa
2 - Sympa sends a command report to X
3 - X has installed a home-made vacation program replying to programs
4 - Sympa processes the reply and sends a report
5 - Looping to step 3
\end{verbatim}
\end {quote}

\Sympa keeps track of each reports sent to an address in an internal counter.
The loop detection algorithm is :

\begin {itemize}

	\item Increment the counter

	\item If we are within the sampling period (as defined by
	\cfkeyword {loop\_command\_sampling\_delay} parameter)

	\begin {itemize}
		\item Then, if the counter exceeds the 
		\cfkeyword {loop\_command\_max} parameter then 
		don't send the report and notify the listmaster

		\item Else, start a new sampling period and reinitialize
		the counter,  ie multiply it by the 
		\cfkeyword {loop\_command\_decrease\_factor} parameter
	\end {itemize}


\end {itemize}
%%%%%%%%%%%%%%%%%%%%%%%%%%%%%%%%%%%%%%%%%%%%%%%%%%%%%%%%%%%%%%%%%%%%%%%%%%%%%%
% Mailing lists definition
%%%%%%%%%%%%%%%%%%%%%%%%%%%%%%%%%%%%%%%%%%%%%%%%%%%%%%%%%%%%%%%%%%%%%%%%%%%%%%

\cleardoublepage
\chapter {Mailing list definition}
    \label {ml-creation}

The mailling list creation tool is Sympa's web interface. However this
web feature is available only from 2.7 release. Previous  version users
have to create new lists with there favorit text file editor.  

This chapter describe how to create mailing list without using
web tools. See~\ref {web-ml-creation}, page~\pageref {web-ml-creation} to
do it using WWSympa, you will do it in a easier way.

To create a mailing list, only system mail aliases modification requires
system privileges. Other tasks will be carried out under \texttt
{sympa} UID in order to guarantee that the files created have the
right permissions.


\begin {itemize}
        \item add aliases in alias file

        \item create the list directory  \tildedir {sympa/expl/\samplelist}

        \item create configuration file in \tildedir
            {sympa/expl/\samplelist} directory

        \item create  customized message files (welcome, bye, removed
          remind, message.header, message.footer) if needed ; you probably need at least the
          welcome.

\end {itemize}


\section {Mail aliases}
    \label {list-aliases}
    \index{aliases}
    \index{mail aliases}

For each new list, it is necessary to create three mail aliases
(the location of the \unixcmd {sendmail} alias file varies from
one system to another).

For example, to create the \mailaddr {\samplelist} list, the following
aliases must be added:

\begin {quote}
    \tt
    \begin {tabular} {ll}
        \mailaddr {\samplelist}:         &
            "|/home/sympa/bin/queue \samplelist"
            \\
        \mailaddr {\samplelist-request}: &
            "|/home/sympa/bin/queue \samplelist-request"
            \\
        \mailaddr {\samplelist-editor}:  &
            "|/home/sympa/bin/queue \samplelist-editor"
            \\
        \mailaddr {\samplelist-owner}:   &
            "|/home/sympa/bin/bouncequeue \samplelist
            \\
        \mailaddr {\samplelist-subscribe}:   &
            "|/home/sympa/bin/queue \samplelist-subscribe"
            \\
        \mailaddr {\samplelist-unsubscribe}: &
            "|/home/sympa/bin/queue \samplelist-unsubscribe"
            \\

    \end {tabular}
\end {quote}

%This example demonstrates how to define a list with the low priority
%level 2. Messages for editor and owner will be processed by \Sympa
%with greater priority (level 1) than messages to the list itself.

The address \mailaddr {\samplelist-request} should be the address
of the person in charge of \mailaddr {\samplelist} list management
(the list \textindex {owner}).  \Sympa will forward messages for
\mailaddr {\samplelist-request} to the owner of list \mailaddr
{\samplelist} defined in \tildefile {sympa/expl/\samplelist/config}
file.  This feature avoids later changes to this alias when updating
owner's addresses.

The address \mailaddr {\samplelist-editor} can be used as \mailaddr
{\samplelist-request} to contact the list editors if defined in
\tildefile {sympa/expl/\samplelist/config}.  This address definition
is not mandatory.

The address \mailaddr {\samplelist-owner} is the address receiving
non-delivery reports. \file {bouncequeue} program stores these messages 
in \dir {queuebounce} directory. \WWSympa ((see~\ref {wwsympa}, page~\pageref {wwsympa}) may then analyze them and provide a web access to them.

The address \mailaddr {\samplelist-subscribe} is for a fast and
easy to explain subscription. Be carefull: subscription is so easy
that spammer may subscribe by accident.

The address \mailaddr {\samplelist-unsubscribe} is for simple
unsubscription. By the way, everything that simplifies unsubscription
simplifies list administration.


\section {List directory}
\label {list-directory}

Each list has its own directory which name defines the list name. We
recommend to create it with the same name as the alias. This directory is
located in the \tildedir {sympa/expl} according to \cfkeyword{home}
defined in \file {/etc/sympa.conf} file.

Here is a list of files/directories you may find in the list's directory :

\begin {quote}
\begin{verbatim}
archives/
bye.tpl
config
info
invite.tpl
homepage
message.header
message.footer
reject.tpl
remind.tpl
removed.tpl
stats
subscribers
welcome.tpl
\end{verbatim}
\end {quote}

\section {List configuration file}
    \label {exp-config}


The configuration file for the \mailaddr {\samplelist} list is named
\tildefile {sympa/expl/\samplelist/config}. \Sympa reads it into memory
the first time the list is referred to. This file is not rewritten by 
\Sympa; as a result, it is possible to put comment lines in it. 
It is possible to change this file when the program is running. 
Changes are then taken into account the next time the list is
accessed. Be careful to provide read access for \Sympa on this file !

You will find a few configuration files in \dir {sample} directory. Copy
one of them to \tildefile {sympa/expl/\samplelist/config} and customize it.

List configuration parameters are reviewed in list creation section, \ref {list-configuration-param}, page~\pageref {list-configuration-param}.

\section {Examples of configuration files}

This first example is for a list open to everyone:

\begin {quote}
\begin{verbatim}
subject First example (an open list)

visibility noconceal

owner
email Pierre.David@prism.uvsq.fr

send public

review public
\end{verbatim}
\end {quote}

The second example is for a moderated list with authentified subscription:
\index{moderation}
\index{authentication}

\begin {quote}
\begin{verbatim}
subject Second example (a moderated list)

visibility noconceal

owner
email moi@ici.fr

editor
email big.prof@ailleurs.edu

send editor

subscribe auth

review owner

reply_to list

cookie 142cleliste
\end{verbatim}
\end {quote}

The third example is for a moderated list, with subscription
controlled by the owner, and running in digest mode. Subscribers
who are in \textindex {digest} mode receive the mail on Monday and
Thursday.

\begin {quote}
\begin{verbatim}
owner
email moi@ici.fr

editor
email prof@ailleurs.edu

send editor

subscribe owner

review owner

reply\_to list

digest 1,4 12:00
\end{verbatim}
\end {quote}

\section {Subscribers file}
    \label {file-subscribers}
    \index{subscriber file}

\textbf {WARNING}: \Sympa will not use this file if the list is configured in \texttt {include} or \texttt {database} \lparam{user\_data\_source}.

The \tildefile {sympa/expl/\samplelist/subscribers} file is created and is
automatically filled in. It contains information about list
subscribers.  It is not advisable to edit this file.  Main parameters
are:

\begin {itemize}
    \item \lparam {email} \textit {address}

        Email address of subscriber.

    \item  \lparam {gecos} \textit {data} 

        Information about subscriber (family name, first name,
        etc.) This parameter is optional on signing up.

    \item \lparam {reception}
            \texttt {nomail} $|$
            \texttt {digest} $|$
            \texttt {summary}
        \label {par-reception} 

        Special receive mode for emails that the subscriber has
        selected. The mode can be either \textit {nomail},  \textit
        {digest} or \textit {summary}.  In normal receive mode, the receive attribute
        no longer appears for this subscriber.  See the \mailcmd
        {SET~LISTNAME~SUMMARY} (\ref {cmd-setsummary}, 
        page~\pageref {cmd-setsummary}) ,
        or \mailcmd {SET~LISTNAME~NOMAIL} command (\ref {cmd-setnomail},
        page~\pageref {cmd-setnomail}), and the \lparam {digest}
        parameter (\ref {par-digest}, page~\pageref {par-digest}).

    \item \lparam {visibility} \texttt {conceal}  
        \label {par-visibility-conceal}

        Special mode which allows subscriber to be invisible when
        there is a \mailcmd {REVIEW} command in the list.  If this
        parameter does not exist, the subscriber remains visible
        by \mailcmd {REVIEW}.  Note: this option does not affect
        the results of a \mailcmd {REVIEW} command issued by an
        owner.  See the \mailcmd {SET~LISTNAME~MAIL} command (\ref
        {cmd-setconceal}, page~\pageref {cmd-setconceal}) for
        details.

\end {itemize}


\section {Info file}

\tildefile {sympa/expl/\samplelist/info} should contain a detailed text
description of the list. If is used by \mailcmd {INFO} command. 
It can also be referenced from template files for service messages.

\section {Homepage file}

\tildefile {sympa/expl/\samplelist/homepage} is the HTML text 
on the \WWSympa info page for the list.

\section {List template files}
\label{list-tpl}
\index{templates, list}

These files are used by Sympa as service messages for commands such as
\mailcmd {SUB}, \mailcmd {ADD}, \mailcmd {SIG}, \mailcmd {DEL}, \mailcmd {REJECT}. 
These files are interpreted (parsed) by \Sympa and respect the template 
format ; every file has the .tpl extension. See \ref {tpl-format}, 
page~\pageref {tpl-format}. 

Sympa looks for these files in the following order :
\begin {enumerate}
 	\item \tildedir {sympa/expl/\samplelist/<file>.tpl} 
	\item \tildedir {sympa/etc/templates/<file>.tpl}. 
	\item \tildedir {sympa/bin/etc/templates/<file>.tpl}.
\end {enumerate}

If the file starts with a From: line it is considered as
a full message and will be sent (after parsing) without adding SMTP
headers. Else file is taken as a text/plain message body.

The following variables may be used in list template files :

\begin {itemize}

	\item[-] [conf-$>$email] : sympa email address local part

	\item[-] [conf-$>$host] : sympa host name

	\item[-] [conf-$>$sympa] : sympa's complete email address

	\item[-] [conf-$>$wwsympa\_url] : \WWSympa root URL

	\item[-] [conf-$>$listmaster] : listmaster email addresses

	\item[-] [list-$>$name] : list name

	\item[-] [list-$>$host] : list hostname

	\item[-] [list-$>$lang] : list language

	\item[-] [list-$>$subject] : list subject

	\item[-] [list-$>$owner] : list owners table hash

	\item[-] [user-$>$email] : user email address

	\item[-] [user-$>$gecos] : user gecos field (usually his/her name)

	\item[-] [user-$>$password] : user password

	\item[-] [user-$>$lang] : user language
	

\end {itemize}

You may also dynamically include a file from a template using the
[INCLUDE] directive.


\textit {Example:} 

\begin {quote}
\begin{verbatim}
Dear [user->email],

Welcome to list [list->name.@[list->host].

Presentation of the list :
[INCLUDE 'info']

Owners of liste [list->name] are :
[FOREACH ow IN list->owner]
   [ow->gecos] <[ow->email]>
[END]


\end{verbatim}
\end {quote}

\subsection {welcome.tpl} 

\Sympa will send a welcome message for each subscription. The welcome 
message can be customized for each list.

\subsection {bye.tpl} 

Sympa will send an unsubscription message for each received SIGNOFF 
mail command.

\subsection {removed.tpl} 

This message is sent to users that have been deleted (using \mailcmd {DELETE} 
command) from the list by the list owner.

\subsection {reject.tpl} 

\Sympa will send a reject message to the sender of a message that is rejected
par the list editor. If the editor prefixe the \mailcmd {REJECT} with the
keyword QUIET, the reject message is not sent.

\subsection {invite.tpl} 

This message is sent to users that have been invited (using \mailcmd {INVITE} 
command) to subscribe to a list. 

You may use additional variables
\begin {itemize}

	\item[-] [requested\_by] : email of the person who sent the 
		\mailcmd{INVITE} command

	\item[-] [url] : the mailto: URL to subscribe to the list

\end {itemize}

\subsection {remind.tpl}

This file contains a message which is sent to each subscriber
when one of the list owners sends the command \mailcmd {REMIND}
 (see~\ref {cmd-remind}, page~\pageref {cmd-remind}).

\section {Stats file}
    \label {stats-file}
    \index{statistics}

\tildefile {sympa/expl/\samplelist/stats} is a text file containing 
statistics about the list. These informations are numerics separated
by white spaces within a single line :

\begin {itemize}

	\item Number of messages sent, used to generate X-sequence headers

	\item Number of messages X nb recipients 

	\item Number of bytes X nb messages

	\item Number of bytes X nb messages X nb recipients

	\item Number of subscribers

\end {itemize}

\section {Message header and footer} 
\label {messagefooter}

You may create \tildefile {sympa/expl/\samplelist/message.header} and
\tildefile {sympa/expl/\samplelist/message.footer} files. Their content
is added, either at the beginning or at the end of each message 
before the distribution process. 

\lparam {footer\_type} list parameter defines whether to attach the 
header/footer content as a MIME part (except for multipart/alternative 
messages) or to append them to the message body (if a text/plain message).

\subsection {Archives directory}    

\tildedir {sympa/expl/\samplelist/archives/} directory contains the 
archived  messages if the list is archived, see \ref {par-archive}, 
page~\pageref {par-archive}. The files are named in accordance with the 
frequency of archiving defined with the \lparam {archive} parameter.



%%%%%%%%%%%%%%%%%%%%%%%%%%%%%%%%%%%%%%%%%%%%%%%%%%%%%%%%%%%%%%%%%%%%%%%
% List configuration parameters
%%%%%%%%%%%%%%%%%%%%%%%%%%%%%%%%%%%%%%%%%%%%%%%%%%%%%%%%%%%%%%%%%%%%%%%

\cleardoublepage
\chapter {Creating and editing mailing using the web}
    \label {web-ml-creation}

The mailing list management by list owners process is designed for
web interface. It is based on strict definition of privileges of
listmaster, main list owner and basic list owner. The goal is to
allow each listmaster to define who can create a list and what
parameter can be set by owners. Therefore, a complete
installation need some effort on the configaration side, though defaults
are acceptable for most sites.

Some features are allready available, others will be soon, this is mentioned 
in this documention.

\section {List creation}

Lismasters have all privileges. Currently it
is defined in \file {sympa.conf} but in the future, it might be possible to
define one listmaster per virtual robot. As the default behaviour, newly created
lists are activated by listmaster ; list creation is available to any intranet user 
(ie : users with an email address within the same domain as Sympa).
This is controlled by \cfkeyword {create\_list} scenario.

\subsection {create\_list scenario}

This scenario can apply any condition about the [sender]
(ie: WWSympa user), and it must return \cfkeyword {reject}, \cfkeyword {do\_it}
or \cfkeyword {listmaster}.
The "create" button is available from the main menu only if the user is
allowed by the create\_list scenario.
If the scenario returns \cfkeyword {do\_it}, the list will be created and installed ;
if it returns "listmaster", the user is allowed to create a list but
the list is created with the \cfkeyword {pending} status
and only owner of the list can view or use it. In order to
install or reject new list, the listmaster need to open the list of pending lists
using the "pending list" button on the "server admin"
menu. Then he can install or reject pending lists.

\subsection {typical list profile}

Many different usages are possible for mailing list. \Sympa proposes a
large set of parameters for each list in order to adapt the list's behaviour
to any kind of usage. It may be difficult for users to choose any of the
list parameters to define a list profile. So the create list form just asks
the list creator to choose a profile for his/her list and complete it with
owner email, list subject and short list description.

The list profiles can be stored in \tildedir {sympa/etc/create\_list\_templates} or
\tildedir {sympa/bin/etc/create\_list\_templates} which come with Sympa's
distribution and should not be modified.  
\tildedir {sympa/etc/create\_list\_templates} which will not be
overwritten by make install and is intended to contain site customization.


A list profile is nearly a list configuration but some fields such as owner email
will be replaced by WWSympa et installation time. It is easy to create new list 
templates by modifying an existing one. Contributions to the distribution are welcome.

You might want to hide or modify profiles (unusefull or dangerous 
for your site). If a profile exists both in the local site directory
\tildedir {sympa/etc/create\_list\_templates} and
\tildedir {sympa/bin/etc/create\_list\_templates} directory, then the local profile 
will be used by WWSympa. 

Another way to control publically available profiles is to
edit \cfkeyword {create\_list.conf} file (the default for this file is in
\tildedir {sympa/bin/etc/} directory and you can create one in \tildedir {sympa/etc/}).
This allows to list available list templates. Example :
\begin {quote}
\begin{verbatim}
# Dont allow the public_anonymous profile
public_anonymous hidden
* read
\end{verbatim}
\end{quote}


When a list is created, whatever is it's status \cfkeyword {pending} or
\cfkeyword {open}, the owner can use WWSympa admin features to modify the list
parameters (in futur version) or to edit welcome message and so on.

WSYympa keeps creation and modification authors and date in the list's
\file {config} file (previous configuration file are saved).

\subsection {creating list alias}
 Managing alias need root access and is depending on the local mail organization.
So it'is not for sympa application tomodify and install aliases. We plans to
introduce a plugin solution so anyone can write its own alias generation. Not
yet available.

\section {List edition}
\label {list-edition}

For each parameter, you can choose via \tildefile {sympa/etc/edit\_list.conf}
configuration file who can edit list parameters ; the default 
\tildefile {sympa/bin/etc/edit\_list.conf} is reasonnably safe.

\begin {quote}
\begin{verbatim}
     examples :

         \# only listmaster can edit user\_data\_source, priority, ...
         user\_data\_source listmaster  

         priority listmaster
      
         \# only privileged owner can modify  editor parameter, send, ...
         editor privileged\_owner

         send privileged\_owner

         \# other parameters can be changed by simple owners
         default owner
\end{verbatim}
\end {quote}

      Privileged owners are defined in list's \file {config} file as follows :
	\begin {quote}
	\begin{verbatim}
		owner
		email his@email
      		profile privileged
	\end{verbatim}
	\end {quote}

      The following rules are hardcoded in wwsympa :
\begin {itemize}

\item listmaster is privileged owner of any list 

\item listmaster is the only one who can edit the "profile privileged"
      owner attribut. 

\item owner can edit there own attibut (except profile and email)

\item when creating a new list the requestor is privileged owner

\item privileged owner can edit any gecos,reception,info attribut
of any owner

\item privileged owner can edit owner's emails (unless the owner he is
editing is privileged)

\end {itemize}

      The goal is to define two trust level for owners (some can just 
      edit gadget like "custom\_subject", some others can edit important
      parameters, but keep control of crutial parameters like 
      the list of privileged owners or user\_data\_sources. So owners can change
      there e-mail (asking to a privileged owner to do it), but they can't
      give away the list mangement responsability to someone else without
      refering to listmaster.




%%%%%%%%%%%%%%%%%%%%%%%%%%%%%%%%%%%%%%%%%%%%%%%%%%%%%%%%%%%%%%%%%%%%%%%
% List configuration parameters
%%%%%%%%%%%%%%%%%%%%%%%%%%%%%%%%%%%%%%%%%%%%%%%%%%%%%%%%%%%%%%%%%%%%%%%

\cleardoublepage
\chapter {List configuration parameters}
    \label {list-configuration-param}


The configuration file is made up of paragraphs separated by empty
lines and introduced by a keyword.

% [sa] Incomplet, �num�ration mal pr�sent�e et non cliquable
% [pda] : c'est maintenant cliquable. Pour la pr�sentation, on verra plus tard, lorsque ce sera complet

However the number of parameters is important, the minimal list
definition is very short, only  \lparam {owner} and \lparam {subject}
parameter are needed  because all the other parameter have a default value.

\begin {quote}
    \textit {keyword value}
\end {quote}

\textbf {WARNING}: configuration parameters must be separated by
empty lines and EMPTY LINES ONLY !

\section {List description}

\subsection {editor}
    \label {par-editor}
    \index{moderation}

The \file {config} file contains one \lparam {editor} paragraph
per \textindex {moderator} (or editor).

\textit {Example:} 

\begin {quote}
\begin{verbatim}
editor
email Pierre.David@prism.uvsq.fr
gecos Pierre (Universit� de Versaille St Quentin)
\end{verbatim}
\end {quote}

The editor of a list is the only one authorized to send messages
in the list when the \lparam {send} parameter (see~\ref {par-send},
page~\pageref {par-send}) is set to \lparam {editor}.\lparam
{editorkey} or \lparam {editorkeyonly}
This parameter is used also to redistribute mail in some other situation
( \lparam {privateoreditorkey} ).

The syntax of this directive is the same as that of the \lparam
{owner} parameter (see~\ref {par-owner}, page~\pageref {par-owner}),
even when several moderators are defined.

\subsection {host}
 \label {par-host}
 \index{host}

	\default {\cfkeyword {host} robot parameter}

\lparam {host} \textit {fully-qualified-domain-name}

Domain name hosting the list (used for \rfcheader {From} , for
example).Default value is set in
 \file {/etc/sympa.conf} file.

\subsection {lang}
    \label {par-lang}

	\default {\cfkeyword {lang} robot parameter}

\textit {Example:} 

\begin {quote}
\begin{verbatim}
\lparam {lang} cn-big5
\end{verbatim}
\end {quote}

This parameter defines the language used for the list. It is
exploited to initialize user's lang preference ; \Sympa command
reports are extracted from the associated messages catalog.

See \ref {internationalization}, page~\pageref {internationalization}
for available langages.

\subsection {owner}
    \label {par-owner}


The \file {config} file contains one \lparam {owner} paragraph per owner. 

\textit {Example:} 

\begin {quote}
\begin{verbatim}
owner
email serge.aumont@cru.fr
gecos C.R.U.
info Tel: 02 99 76 45 34
reception nomail
\end{verbatim}
\end {quote}

The list owner is usually the person who has the authorization to send
\mailcmd {ADD} (see~\ref {cmd-add}, page~\pageref {cmd-add}) and
\mailcmd {DELETE} (see~\ref {cmd-delete}, page~\pageref {cmd-delete})
commands on behalf of other users.

When the \lparam {subscribe} parameter (see~\ref {par-subscribe},
page~\pageref {par-subscribe}) specifies a restricted list, the
owner is the only one who has the capability to subscribe users.
In this case, \mailcmd {SUBSCRIBE} requests are forwarded to him.

There may be several owners of a single list; in this case, each
owner is defined by a paragraph introduced by the \lparam {owner}
keyword.

The \lparam {owner} directive is followed by one or several lines
giving details regarding the owner's characteristics:

\begin {itemize}
    \item  \lparam {email} \textit {address}

        Owner's email address

    \item  \lparam {reception nomail}

        Optional attribute for an owner who don't want to receive
        mails.  Useful to define an owner with multiple email
        addresses: they are all recognized when \Sympa receives
        mail, but only some of them are recipient of administrative
        mail from \Sympa.

    \item  \lparam {gecos} \textit {data}

        Public information on the owner

    \item \lparam {info} \textit {data}

	Available since release 2.3

	Private information on the owner

    \item \lparam {profile} \texttt {privileged} $|$
	                    \texttt {normal}

	Available since release 2.3.5

	Profile of the owner. This is currently used to restrict
	access to some feature in WWSympa such as adding new owners
	to a list.

\end {itemize}

\subsection {subject}
    \label {par-subject}

\lparam {subject} \textit {subject-of-the-list}

This parameter indicates the subject of the list, which is sent in
response to the \mailcmd {LISTS} mail command. The subject is
a free form text limited to one line.
This parameter is \emph {not} used by \Sympa if \tildefile
{sympa/expl/lists} file exists (e.g.  static list of lists).

\subsection {topics}
    \label {par-topics}

\lparam {topics} computing/internet,education/university

This parameter allows classification of lists. You may define multiple 
topics as well as hierarchical ones. \WWSympa's list of public lists 
uses this parameter.

\subsection {visibility }
    \label {par-visibility}

	\default {conceal}

	\scenarized {visibility}

\lparam {visibility}
    \texttt {conceal} $|$
    \texttt {noconceal}

This parameter indicates whether the list should be listed when
replying to \mailcmd {LISTS} command.  This parameter is \emph {not}
used by \Sympa if \tildefile {sympa/expl/lists} file exists
(e.g. static list of lists).


\section {Data source related}

\subsection {user\_data\_source}

    	\label {par-user-data-source}
	\index{user-data-source}

	\default {file|database, if using a DBMS}

\lparam {user\_data\_source}
   \texttt {file} $|$
   \texttt {database} $|$
   \texttt {include}

Sympa allows the mailing list manager to choose how \Sympa loads
subscribers' data. Subscriber's informations can be stored in a text 
file, in a RDBMS or included from various external sources (list,
flat file, LDAP directory, RDBMS).

\begin {itemize}
\item  \lparam {user\_data\_source} \texttt {file}

       When this value is used subscribers data
       are stored in a file which name is define \file
       {sympa.conf}, parameter \cfkeyword {subscribers}.
       This is maintened for backwared compatibility.

\item  \lparam {user\_data\_source} \texttt {database} 

       This mode as been introduced to store data in a RDBMS 
       in order to share subscribers data with a HTTP interface 
       and also to be able to manage very large mailing lists 
       (tested with MySql, 200.000 subscribers in a single list). 
       We strongly recommend to use this value (and of course to
       configure a RDBMS). This improve Sympa performence and solve
       possible conflict with sympa and wwsympa.
        Please refer to the 
       section \"Using \Sympa with a RDBMS\" 
       (\ref {sec-rdbms}, page~\pageref {sec-rdbms}).

\item \lparam {user\_data\_source} \texttt {include} 
       
       Subscribers list is not define by email enumeration 
       but by subscriber properties. Include can be performed 
       by extracting emails using a SQL or LDAP query or 
       by including other mailing lists. Compulsary include 
       paragraphs are needed (at least one), each defining a
       data source. These paragraphs' name can be one of
       \lparam {include\_file}, \lparam {include\_list}, 
	\lparam {include\_sql\_query} or \lparam {include\_ldap\_query}. 
\end {itemize}


\subsection {ttl}

    	\label {ttl}
	\index{ttl}

	\default {3600}

\lparam {ttl} \texttt {delay\_in\_seconds} 

\Sympa caches users data extracted using the include parameter.
The time to leave of these data in \Sympa can be controled with this
parameter. The default value is 3600.

\subsection {include\_list}

    	\label {include-list}
	\index{include-list}

\lparam {include\_list} \texttt {listname}

This parameter will be interpreted only if 
\lparam {user\_data\_source} is set to \texttt {include}.
All subscribers of list \texttt {listname} will become subscribers 
of the current list. You may include as many lists as needed using one
\lparam {include\_list} \texttt {listname} line for each included
list. You can include any list whatever is
its  \lparam {user\_data\_source} definition. In particular, you can
include lists which are also defined by inclusion of other lists. 
Be carefull not to include list \texttt {A} in list \texttt {B} and
then list \texttt {B} in list \texttt {A} : this would start an 
infinite loop.


\subsection {include\_sql\_query}
    \label {include-sql-query}

\lparam {include\_sql\_query}

This parameter will be interpreted only if the
\lparam {user\_data\_source} value is set to  \texttt {include}, it is
the begining of a paragraph which defines the SQL query parameters :

\begin{itemize}

\item
\label {db-type}
\lparam {db\_type} \textit {dbd\_name} 

This is the DBD name (Oracle, mysql , Pg, Sybase ...). You must respect the case.

\item
\label {host}
\lparam {host} \textit {hostname}

The Database Server \Sympa will try to connect to.

\item
\label {db-name}
\lparam {db\_name} \textit {sympa\_db\_name}

The hostname of the database system.

\item 
\label {user}
\lparam {user} \textit {user\_id}

The user id to connect to the database.

\item 
\label {passwd}
\lparam {passwd} \textit {some secret}

The user passwd for \lparam {user}.


\item
\label {sql-query}
\lparam {sql\_query} \textit {a query string}
The sql query string. Of course, this query must select e-mail, only
e-mail !


\end{itemize}

Example :

\begin {quote}
\begin{verbatim}

user_data_source include

include_sql_query
      db_type oracle
      host sqlserv.admin.univ-x.fr
      user banalise
      passwd mysecret
      db_name scolarship
      sql_query SELECT DISTINCT email FROM student

\end{verbatim}
\end{quote}

\subsection {include\_ldap\_query}
    \label {include-ldap-query}

\lparam {include\_ldap\_query}

This paragraph defines parameters for a LDAP query returning a
list of subscribers. This paragraph is used only if \lparam
{user\_data\_source} is set to \texttt {include}. This feature
requires Net::LDAP (perlldap) Perl module.

\begin{itemize}

\item
\label {host}
\lparam {host} \textit {ldap\_directory\_hostname} 

Name of the LDAP directory host.

\item
\label {port}
\lparam {port} \textit {ldap\_directory\_port} (Default 389) 

Port on which the Directory is accepting connections.

\item
\label {user}
\lparam {user} \textit {ldap\_user\_name}

Username with read access on the LDAP directory.

\item
\label {passwd}
\lparam {passwd} \textit {LDAP\_user\_password}

Password for \lparam {user}.


\item
\label {suffix}
\lparam {suffix} \textit {directeory name}

Defines the naming space covered by the search (optional, depending on
the LDAP server).

\item
\label {filter}
\lparam {filter} \textit {search\_filter}

Defines the LDAP search filter (RFC 2254 compliant).

\item
\label {attrs}
\lparam {attrs} \textit {mail\_attribute} 
\default {mail}

The attribute containing the email address(es) in the returned object.

\item
\label {select}
\lparam {select} \textit {first $|$ all}
\default {first}

If the query returns multiple values, do we keep the first one or all 
of them.

\end{itemize}

Example :

\begin{quote}
\begin{verbatim}

    include_ldap_query
    host ldap.cru.fr
    suffix dc=cru, dc=fr
    filter (&(cn=aumont) (c=fr))
    attrs mail
    select first

\end{verbatim}
\end{quote}


\subsection {include\_file}
    \label {include-file}

\lparam {include\_file} \texttt {path to file} 

This parameter will be interpreted only if the
\lparam {user\_data\_source} value is set to  \texttt {include},
the file is interpreted as one email per line (line beginning with a
"\#" are ignored).

\section {Commands' behaviour}

\subsection {subscribe}
    \label {par-subscribe}

	\default {open}

	\scenarized {subscribe}

Subscribe parameter define the rules for subscribing to the list. 
Predefined scenari are :

\lparam {subscribe}
    \texttt {open} $|$
    \texttt {owner} $|$
    \texttt {closed} $|$
    \texttt {auth} $|$


\begin {itemize}
    \item \lparam {subscribe} \texttt {open}[\texttt {\_notify}]

          Anyone can join the list with the \mailcmd {SUBSCRIBE}
          mail command. Optional notification of each succesful
          subscription to owners can be set for each list with:

          \texttt {subscribe open\_notify}.

    \item \lparam {subscribe} \texttt {owner}

          Only the owner (see \lparam {owner} above) can subscribe
          or unsubscribe a user. The owner should send an \mailcmd
          {ADD} mail command to subscribe the user.

    \item \lparam {subscribe} \texttt {closed}

          Any subscription is refused, except indirectly by the
          owner using the \mailcmd {ADD} command (see~\ref {cmd-add},
          page~\pageref {cmd-add}). This parameter is forced to
          \texttt {closed} value for lists defined using
          \lparam {user\_data\_source}\texttt {include}.
          

    \item \lparam {subscribe} \texttt {auth}[\texttt {\_notify}]

        Use of the \mailcmd {SUBSCRIBE} command requires prior
        \textindex {authentication} of the origin of the subscription
        request. \Sympa sends a key to the requesting party, who
        should in return send \Sympa a \mailcmd {SUBSCRIBE} command
        prefixed with \lparam {auth}~\textit {key}.

        Note: The \lparam {cookie} parameter (see~\ref {par-cookie},
        page~\pageref {par-cookie}) allows allocation of a single
        key.


\end {itemize}

When the optional \texttt {notify} parameter option is set, owners receive
a notification of each subscription, unless their owner definition
include the \texttt {nomail} option.

\subsection {unsubscribe}
    \label {par-unsubscribe}

	\default {open}

	\scenarized {unsubscribe}

This parameter specifies the unsubscription method for the list.
Use \texttt {open\_notify} or \texttt {auth\_notify} to allow owner
notification of each unsubscribe command. 
Predefined scenari are :


\lparam {unsubscribe}
    \texttt {open}[\texttt {\_notify}] $|$
    \texttt {auth}[\texttt {\_notify}]
    \texttt {owner}
    \texttt {closed}

\begin {itemize}
    \item \lparam {unsubscribe} \texttt {open}

          Anyone can unsubscribe from the list.

    \item \lparam {unsubscribe} \texttt {auth}

        Use of the \mailcmd {SIGNOFF} command requires prior
        \textindex {authentication}
        of the sender. To perform this authentification, \Sympa
        sends a key, indicating that the ``\lparam {auth}~\textit
        {key}'' parameter should be put in front of the command.

        Note: it is advisable to enter the \lparam {cookie} parameter
        (see~\ref {par-cookie}, page~\pageref {par-cookie}).

    \item \lparam {unsubscribe} \texttt {closed}
      
        Unsubscription is not allowed. This parameter is forced to
        \texttt {closed} value for lists defined using \lparam
        {user\_data\_source}~\texttt {include}.

\end {itemize}

\subsection {add}
    \label {par-add}

	\default {owner}

	\scenarized {add}

This parameter specifies who can perform \mailcmd {ADD} command
Predefined scenari are :


\lparam {add}
    \texttt {auth} $|$
    \texttt {owner}  $|$
    \texttt {closed}

\begin {itemize}
    \item \lparam {add} \texttt {owner}

          Only owners can add subscribers to the list.

    \item \lparam {add} \texttt {auth}

        Use of the \mailcmd {ADD} command requires prior
        \textindex {authentication}
        of the owner. To perform this authentification, \Sympa
        sends a key, indicating that the ``\lparam {auth}~\textit
        {key}'' parameter should be put in front of the command.

    \item \lparam {unsubscribe} \texttt {closed}
      
         \mailcmd {ADD} is not allowed. This parameter is forced to
        \texttt {closed} value for lists defined using \lparam
        {user\_data\_source}~\texttt {include}.

\end {itemize}

\subsection {del}
    \label {par-del}

	\default {owner}

	\scenarized {del}

This parameter specifies who can perform \mailcmd {DEL} command
Predefined scenari are :


\lparam {del}
    \texttt {auth} $|$
    \texttt {owner}  $|$
    \texttt {closed}

\begin {itemize}
    \item \lparam {del} \texttt {owner}

          Only owners can delete subscribers of the list.

    \item \lparam {DEL} \texttt {auth}

        Use of the \mailcmd {DEL} command requires prior
        \textindex {authentication}
        of the owner. To perform this authentification, \Sympa
        sends a key, indicating that the ``\lparam {auth}~\textit
        {key}'' parameter should be put in front of the command.

    \item \lparam {DEL} \texttt {closed}
      
         \mailcmd {DEL} is not allowed. This parameter is forced to
        \texttt {closed} value for lists defined using \lparam
        {user\_data\_source}~\texttt {include}.

\end {itemize}


\subsection {remind}
    \label {par-remind}

	\default {owner}

	\scenarized {remind}

This parameter specifies who can perform \mailcmd {remind} command
Predefined scenari are :


\lparam {remind}
    \texttt {owner}  $|$
    \texttt {listmaster}

\begin {itemize}
    \item \lparam {remind} \texttt {owner}

          Only owners can perform \mailcmd {remind} for the list.

    \item \lparam {remind} \texttt {listmaster}

        Use of the \mailcmd {remind} is a listmaster privilege.

\end {itemize}


\subsection {send}
    \label {par-send}

	\default {private}

	\scenarized {send}

Predifined scenarii are :

    \texttt {public} $|$
    \texttt {private} $|$
    \texttt {privateoreditorkey} $|$
    \texttt {privateorpublickey} $|$
    \texttt {editor} $|$
    \texttt {editorkey} $|$
    \texttt {editorkeyonly}

This parameter specifies who can send messages to the list ; it is performed by scenari :

\begin {itemize}
    \item \lparam {send} \texttt {public} 

        Anyone can send a message to the list, including non-subscribers.

    \item \lparam {send} \texttt {publickey} 

        Anyone can send a message to the list, including non-subscribers;
        \textindex {authentication} is automatically requested by return mail.

    \item \lparam {send} \texttt {private}

        Only subscribers or owners can send a message.

    \item \lparam {send} \texttt {privatekey}

        Only subscribers or owners can send a message; \textindex
        {authentication}
        is systematically requested by return mail.

    \item \lparam {send} \texttt {privateorpublickey} 

        Anyone can send a message to the list. Authentication is
        requested by return mail for non-subscribers.

    \item \lparam {send} \texttt {privateoreditorkey} 

        Anyone can send a message to the list. Message for which
        the sender is not a subscriber are sent to the list editor
        to confirm or reject its distribution.

    \item \lparam {send} \texttt {editorkeyonly}

        All messages are stored on the server awaiting approval
        for distribution. An enabling key is sent to the moderators
        wit a copy of the message. The message is distributed when
        the \Sympa robot receives a \mailcmd {DISTRIBUTE} command
        with the sole key referring to the queued message.

        The \mailcmd {REJECT} command delete the message from the
        \textindex {moderation} spool (see \cfkeyword {queuemod}
        configuration variable).

        This method is used for:

        \begin {itemize}
            \item rather sure identification of the \textindex
                {moderator}

            \item total independence of moderation mode in
                relation to mail user agent (MUA) type used by
                moderators

            \item Minimum cooperation between moderators in the
                case of sharing of this responsibility.

        \end {itemize}

        With this method, it is not possible for the \textindex
        {moderator} to change the message, which is always refused
        or distributed as a whole. This moderation feature is only
        available as of release 1.2.0 of \Sympa.

    \item \lparam {send} \texttt {editor}

        In this configuration, \Sympa distributes the messages for
        which at least one of the SMTP fields \rfcheader {From},
        \rfcheader {X-Sender} contains
        the address of one of the moderators, distributing these
        in the list (see \lparam {editor} parameter for list
        description file, \ref {par-editor}, page~\pageref
        {par-editor}).

        In the opposite case, it sends these list moderators a
        message entitled ``\textit {Article to moderate}'' containing
        a copy of the article to be distributed. It is up to the
        \textindex {moderator} to send this copy of the message to
        be distributed back to the list, using the redirect,
        redistribute, forward or other functions from its MUA
        (moderation depends on processing of headers, managed by
        each MUA).

        It should be noted that the security level of this method
        is associated to the enabling level of the message SMTP
        fields; it is therefore rather low.

    \item \lparam {send} \texttt {editorkey}

        This method mixes \lparam {send editor} and \lparam {send
        editorkeyonly} functionalities: If one of the SMTP fields
        \rfcheader {From}, \rfcheader {X-Sender} or \rfcheader
        {Approved} contains the address of one of the moderators,
        the message is distributed without any control.

        In the opposite case, it works in the same way as a moderated
        list with \lparam {send editorkeyonly}.

        Available as of release 1.2.0 of \Sympa.

\end {itemize}

\subsection {review}
    \label {par-review}

	\default {owner}

	\scenarized {review}

This parameter specifies who can use the
\mailcmd {REVIEW} (see~\ref {cmd-review}, page~\pageref {cmd-review}),
administrative requests. 
Predifined scenari are :

\lparam {review}
    \texttt {public} $|$
    \texttt {private} $|$
    \texttt {owner} $|$


\begin {itemize}
    \item \lparam {review} \texttt {public}

        \mailcmd {REVIEW} access is not protected and anyone, even
        non-subscribers, can use it.

    \item \lparam {review} \texttt {private}

        The \mailcmd {REVIEW} command is only authorized for
        subscribers.

    \item \lparam {review} \texttt {owner}

        Only the list owner can use it.

\end {itemize}
\subsection {shared documents}
    \label {par-shared}
    \index{shared}

\subsubsection {Read access}

	\default {private}

This parameter specifies who can read shared documents 
the \dir {shared} directory of the list (list what it
contains).

Predifined scenari are :

\lparam {d\_read}
    \texttt {public} $|$
    \texttt {private} $|$
    \texttt {owner} $|$


\begin {itemize}
    \item \lparam {d\_read} \texttt {public}

        \mailcmd {READ} access is not protected and anyone, even
        non-subscribers, can read the directory \dir {shared}.

    \item \lparam {d\_read} \texttt {private}

        Reading the directory \dir {shared} is only authorized for
        subscribers, list owner and listmater.

    \item \lparam {d\_read} \texttt {owner}

        Only the list owner and the listmaster can read the directory \dir {shared}.

\end {itemize}

\subsubsection {Edit access}

	\default {owner}

This parameter specifies who can edit  
in the \dir {shared} directory of the list (who can upload files,
and create subdirectories).

Predifined scenari are :

\lparam {d\_edit}
    \texttt {public} $|$
    \texttt {private} $|$
    \texttt {owner} $|$


\begin {itemize}
    \item \lparam {d\_edit} \texttt {public}

        \mailcmd {EDIT} access is not protected and anyone, even
        non-subscribers, can edit in the directory \dir {shared}.

    \item \lparam {d\_read} \texttt {private}

        Editing the directory \dir {shared} is only authorized for
        subscribers, list owner and listmater.

    \item \lparam {d\_read} \texttt {owner}

        Only the list owner and the listmaster can edit in the directory \dir {shared}.

\end {itemize}

\section {List tuning}

\subsection {reply\_to}
    \label {par-reply-to}

	\default {sender}

\lparam {reply\_to}
    \texttt {sender} $|$
    \texttt {list}   $|$
    \texttt {email}

This parameter indicates whether the \rfcheader {Reply-To} field
should point to the sender of the message (\texttt {sender}) or to
the list itself (\texttt {list}), to anyother e-mail. If \rfcheader {Reply-To} SMTP header field is set in incoming message, \Sympa will not modify it.
If \lparam {forced\_reply\_to} is set, \lparam {reply\_to} is ignored by \Sympa.

Note: it is unadvisable to change this field, especially to point
to the list. Experience demonstrates that numerous users, thinking
they are responding to the sender, send private messages to a list.
This can lead to an embarrassing situation.


\subsection {forced\_reply\_to}
    \label {par-forced-reply-to}

\lparam {forced\_reply\_to}
    \texttt {sender} $|$
    \texttt {list}   $|$
    \texttt {email}

This parameter is like the  \lparam {reply\_to} parameter but using 
this parameter \Sympa overrights any existing rfcheader {Reply-To} SMTP
header.


\subsection {max\_size}
 \label {par-max-size}
 \index{max-size}

	\default {\cfkeyword {max\_size} robot parameter}

\lparam {max\_size} \textit {number-of-bytes}

Maximum size of a message in 8-bit bytes. Default value is set in
 \file {/etc/sympa.conf} file.


\subsection {anonymous\_sender}
    	\label {par-anonymous-sender}
    	\index{anonymous\_sender}

	\lparam {anonymous\_sender} \textit {value}

If this parameter is set for a list, all messages distributed in that list are
anonymized. The SMTP \texttt {From:} fields are replace by \lparam {anonymous\_sender}).
value. Various other field are removed (\texttt {Received:, Reply-To:, Sender: 
X-Sender:, Message-id:, Resent-From:}

\subsection {custom\_header}
    	\label {par-custom-header}
    	\index{custom-header}

	\lparam {custom\_header} \textit {header-field}\texttt {:} \textit {value}

This parameter is optional. The header specified with this parameter
will be placed in each of the message headers distributed in the
list. As of release 1.2.2 of \Sympa, it is possible to put several
custom header lines in the configuration file at the same time.

% [pda] : faudrait le mettre en anglais
% [sa] A voir laisse le commentaire
\example {custom\_header X-url: http://www.cru.fr/listes/apropos/sedesabonner.faq.html}.

\subsection {custom\_subject}

	\label {par-custom-subject}
	\index{custom-subject}

	\lparam {custom\_subject} \textit {value}

This parameter is optional. This parameter specifies a string which is
added to the subject of distributed messages (this is intented to help
users who do not use automatic tools to sort incoming messages).

\example {custom\_subject [sympa-users]}.

\subsection {footer\_type}
    	\label {par-footer-type}
	\index{footer-type}

	\default {mime}

\lparam {footer\_type (optional, default value is mime)}
   \texttt {mime} $|$
   \texttt {append}

List owners may decide to add message header or footer to messages
sent to the list. This parameter defines the way a footer/header is
added to a message.

\begin {itemize}
\item  \lparam {footer\_type} \texttt {mime}

       This is the default value. Sympa will add properly the
       footer/header as a new MIME part. If message if
       multipart/alternative, nothing is added (would require another
       level of MIME encapsulation).


\item  \lparam {footer\_type} \texttt {append} 

        In this configuration, sympa won't create new MIME parts but
        will try to append the header/footer to the body of the
        message. \tildefile
        {sympa/expl/\samplelist/message.footer.mime} will be
        ignored. Header/footer may be appendeded to text/plain
        messages only.


\end {itemize}

\subsection {digest}

    	\label {par-digest}
    	\index{digest}

	\lparam {digest} \textit {daylist} \textit {hour}\texttt {:}\textit {minutes}

Definition of \lparam {digest} mode. If this parameter is filled
in, subscribers can select the receive option in multipart/digest
MIME format.  Messages are then grouped together and this compilation
is sent to the subscribers in accordance with the rythm selected
with this parameter.

\textit {Daylist} designates a list of days in the week in number
format (from 0 for Sunday to 6 for Saturday), separated by commas.

\example {digest 1,2,3,4,5 15:30} 

In this example, \Sympa sends digests at 3:30 PM from Monday to Friday.

\textbf {WARNING}: if the sending time is too late, \Sympa may not
be able to process it. It is essential that \Sympa scans the digest
queue at least once between the time laid down for sending of the
digest and 12:00~PM. Usually do not use digest time later than
11:00~PM.

\subsection {available\_user\_options}

    	\label {par-available-user-options}
	\index{available-user-options}

	\lparam {available\_user\_options} parameter start a paragraph to
	define available options for the subscribers of the list.

\begin {itemize}
   \item \lparam {reception} \textit {modelist}

	\default {\cfkeyword {reception} mail,notice,digest,summary,nomail}

\textit {modelist} is a list of modes (mail, notice, digest, symmary, nomail),
separated by commas. Only these modes will be allowed for the subscribers of
this list. If a subscriber has a reception mode not in the list, sympa uses
the one of \textit {default\_user\_options} paragraph.

\end {itemize}


\subsection {default\_user\_options}

    	\label {par-default-user-options}
	\index{default-user-options}

	\lparam {default\_user\_options} parameter starts a paragraph to
	define a default profile for the subscribers of the list.

\begin {itemize}
    \item \lparam {reception}
            \texttt {notice} $|$
            \texttt {digest} $|$
            \texttt {summary} $|$
            \texttt {nomail} $|$
            \texttt {mail}

        Mail reception mode.

    \item \lparam {visibility}
            \texttt {conceal} $|$
            \texttt {noconceal} 

        Visibility of the subscriber with the \mailcmd {REVIEW}
        command.

\end {itemize}


\subsection {cookie}

    	\label {par-cookie}
	\index{cookie}

	\default {\cfkeyword {cookie} robot parameter}

\lparam {cookie} \textit {random-numbers-or-letters}

This parameter is a secret item for generating \textindex
{authentication} keys for administrative commands (\mailcmd {ADD},
\mailcmd {DELETE}, etc.).  This parameter should remain secret,
even for owners. The cookie is applied to all list owners, and is
only taken into account when the owner has the \lparam {auth}
parameter (\lparam {owner} parameter, see~\ref {par-owner},
page~\pageref {par-owner}).

\example {cookie secret22}

\subsection {priority}
    \label {par-priority}

	\default {\cfkeyword {default\_list\_priority} robot parameter}

\lparam {priority} \textit {0-9}

This is the priority \Sympa will process messages for this list.
This is applied while going through the spool. 

0 is the highest priority. The following priorities can be used:  
\texttt {0...9~z}.
\texttt {z} is a special priority since messages will
stay in spool forever (useful to hangup a list).

Available since release 2.3.1.

\section {bounce}
    \label {bounce}

\lparam {bounce}

This paragraph defines bounce management parameters :

\begin{itemize}


\item
\label {welcome-return-path}
\lparam {welcome\_return\_path} 

	\default {\cfkeyword {welcome\_return\_path} robot parameter}

	If set to \cfkeyword {unique}, the welcome message is sent using
        a unique return path in order to remove the subscriber in case
	of a bounce. 

\item
\label {remind-return-path}
\lparam {remind\_return\_path} 

	\default {\cfkeyword {remind\_return\_path} robot parameter}

	Same as \cfkeyword {welcome\_return\_path} but applyed for remind
        message.

\item
\label {warn-rate}
\lparam {warn\_rate} 

	\default {\cfkeyword {bounce\_warn\_rate} robot parameter}

	Above this rate of bouncing addresses in a list,
	listowner receives a warning for each mail distributed in the list. 

\item
\label {halt-rate}
\lparam {halt\_rate} 

	\default {\cfkeyword {bounce\_halt\_rate} robot parameter}

	\texttt {NOT USED YET}

	If bounce rate reaches the \texttt {halt\_rate}, messages 
	for the list will be halted, ie they are kept for later 
	moderation.Above this rate of bouncing addresses in a list,
	messages for the list aren't distributed. 
	
\end{itemize}

\section {Archives related}

\subsection {archive}
    \label {par-archive}
    \index{archive}

If the \file {config} file contains  a \lparam {archive} paragraph
\Sympa will manage an archive of this list

\textit {Example:} 

\begin {quote}
\begin{verbatim}
archive
period week
access private
\end{verbatim}
\end {quote}

If parameter \lparam {archive} is specified, archives are
accessible to users through the \mailcmd {GET} command, 
and the index of the list archives is given by the \mailcmd {INDEX}
command (last message of a list is available using  \mailcmd {LAST} command).


\lparam {period}
    \texttt {day} $|$
    \texttt {week} $|$
    \texttt {month} $|$
    \texttt {quarter} $|$
    \texttt {year}


This parameter specifies how archiving is organised: by \texttt
{day}, by \texttt {week}, by \texttt {month}, by \texttt {quarter}
or by \texttt {year}.  Generation of automatic list archives requires
creation of an archive directory in the root of the list (\tildedir
{sympa/expl/\samplelist/archives/}) where these documents will be
put.

\lparam {access}
    \texttt {private} $|$
    \texttt {public} $|$
    \texttt {owner} $|$
    \texttt {closed} $|$


This parameter specifies who is allowed to send \mailcmd {GET}, \mailcmd {LAST}  or \mailcmd {INDEX}  commands.



\subsection {web\_archive}
    \label {par-web-archive}
    \index{web\_archive}

If the \file {config} file contains  a \lparam {web\_archive} paragraph
\Sympa will copy all messages ditributed in the list to the
"queueoutgoing"  spool. It is intented to be use with WWSympa html
archive tools. This paragraph must contain at least the access
parameter to control who can browse the web archive.

\textit {Example:} 

\begin {quote}
\begin{verbatim}
web_archive
access private
\end{verbatim}
\end {quote}


Value for web\_archive access parameter must be one of the following :
\lparam {access}
    \texttt {private} $|$
    \texttt {public} $|$
    \texttt {owner} $|$
    \texttt {closed} $|$
    \texttt {listmaster} 

%%%%%%%%%%%%%%%%%%%%%%%%%%%%%%%%%%%%%%%%%%%%%%%%%%%%%%%%%%%%%%%%%%%%%%%%%%%%%%
% Shared documents
%%%%%%%%%%%%%%%%%%%%%%%%%%%%%%%%%%%%%%%%%%%%%%%%%%%%%%%%%%%%%%%%%%%%%%%%%%%%%%

\cleardoublepage
\chapter {Shared documents}
    \label {shared}
    \index{shared}

Shared documents are documents that various users can manipulate
on-line via the web interface of \Sympa, whether they're authorized
to. A shared space belongs to a list, and the users of this list 
can upload, download, delete... documents in the shared space.

\WWSympa shared web features are basic. The goal is not to provide
one more sophisticated tool for web publishing, as web dav or such
product like \textit {Rearsite} provide.
Anyhow, it is very useful to be able to define privilege on
web documents in relation with list attributes, like \textit {subscribers},
\textit {list owners}, or \textit {list editors}. 

All files and directories name are converted by Sympa into lower case. The consequence
is that it is impossible to create two deferent document which name differ only
by char case. This has been introduced in Sympa in order to allow correct URL links
even when using some HTML document generator (typically power point) that use 
random case for file name !

In order to have a better control on the documents of
the shared space, each document is linked with specific information : 
its access rights. The safety is thus ensured.

Shared documents of a list are stored in the directory 
\tildedir {sympa/expl/\samplelist/shared}. 

This chapter describe how the shared documents are managed, 
especially considering access rights control. 
In the following, we will see :  

\begin {itemize}
       	\item the kind of operations performed on shared documents 

        \item the access rights management  

        \item the access rights control specification
	
	\item the actions on shared documents behaviour
        
	\item the template files
\end {itemize}

\section {The three kind of operations on a document}
    \label {shared-operations}
Concerning shared documents, we can distinguish three kind of operations which
have the same constraints relating to access control :
\begin{itemize}
	\item The reading operation :\\
	It consists in :
	\begin{itemize}
		\item If it's a directory, open it and list its content (only the
		sub-documents the user is authorized to ``see'').
		\item If it's a file, download it, but also display it if it's a
		viewable file (\textit {text/plain}, \textit {text/html} or image). 
	\end{itemize}
	\item The edition operation :\\
	It regroups :
		\begin{itemize}
		\item Subdirectory creation	
		\item File uploading
		\item Description of a document (title and basic information)
		\item Edition on-line of a text file
		\item Document (file or directory) removal. 
		If it's a directory, it must be empty.
		\end{itemize}
	These various actions have the same behaviour concerning access rights. If a user is
	authorized to edit a directory, he can create a subdirectory or upload a file inside,
	describe or delete this directory. If a user is authorized to edit a file, he can edit
	it on-line, describe it, replace or remove it.  
	\item The control operation :\\
	The control operation is directly linked with the access rights notion. If we want the
	shared documents to be safe, we have to control the access on document. Everybody must not
	be authorized to do everything on shared documents. Consequently, each document has
	specific access rights for reading and editing. The control on this document consists
	in changing this Read/Edit rights.\\
	The control operation has more restrictive access rights than the two other operations.
	Only the owner of a document, the privileged owner of the list and the listmaster have
	control rights on a document. So, another action possible on a document if you can
	control it is to set the owner of a document.  
\end{itemize}
	


\section {The description file}
\label {shared-desc-file}
The information (title, owner, access rights...) relative to each document must be stocked. That's 
why each shared document is linked with a special file called a description file.
These files have the prefix \file {.desc}.

The description file of a directory having the path \dir {mydirectory/mysubdirectory} has the path
\dir {mydirectory/mysubdirectory/.desc} .
The description file of a file having the path \dir {mydirectory/mysubdirectory/myfile.myextension} has the path
\dir {mydirectory/mysubdirectory/.desc.myfile.myextension} .

\subsection {Structure of description files}

Here is the structure of a decription file of a document : file or directory.
You should \textit {never} have to edit those files.
 
\begin {quote}
\begin{verbatim}
title
  <description of the file in several words>

creation
  email        <email of the owner of the document> 
  date_epoch   <date_epoch of the creation of the document>

access
 read <access rights for read>
 edit <access rights for edit>
\end{verbatim}
\end {quote}

This example is for a document that subscribers can read and only the owner of the document
and the owner of the list can edit.
\begin {quote}
\begin{verbatim}
title
  module C++ which uses the class List

creation
  email foo@some.domain.com
  date_epoch 998698638

access
 read  private
 edit  owner
\end{verbatim}
\end {quote}

\section {The predifined scenarii}
    \label {shared-scenarii}

\subsection {The scenario public}
The scenario \textbf {public} is the most permissive scenario. It enables anyone (including
unknown users) to perform the action.

\subsection {The scenario private}
The scenario \textbf {private} is the basic scenario of a shared space. Every subscriber of
the list is authorized to perform the action. The scenario \textbf {private} is the default
read scenario of the \dir {shared} when the shared space is created. 
The default scenario is used only in the list configuration. You can modify it editing
the list configuration file.

\subsection {The scenario owner}
The scenario \textbf {owner} is the more restrictive scenario of a shared space.
Only the listmaster, the owners of the list, and the owner of the document
or those of one of the parent documents are allowed to perform the action.
The default scenario for editing is the scenario \textbf {owner} in the list
configuration. 

\section {The access control}
    \label {shared-access}
The access control is an important operation, which is performed
before every access to a document of the shared space.

The access control on a document of the hierarchy is an iterative
operation on all its parent directories. 
Here are the regulars adopted for the access control.

\subsection {The listmaster and privileged owner case}
Listmaster and privileged list owner are particular users in the shared
web. They are allowed to perform every action on every document of
the shared space. This precaution permits to keep the control on
the shared space. Listmaster and privileged owner can't be excluded
from any document of the shared space.
 
\subsection {The directory \dir {shared} special case}
In order to allow to have a root directory more restrictive than
the subdirectories, the directory \dir {shared} (root directory) has
a special treatment concerning the access control.
The access rights for read and edit are those of the list configuration file.
The control on the root directory is specific. 
The users allowed to change access rights on directory shared are those allowed to edit the
configuration of the list. 
 
\subsection {General case}
Let's take a document with a relative path in the shared space 
\dir {mydirectory/mysubdirectory/myfile}. This is \textit {not} a document
of the \textit {root} directory. Let's suppose that a user \textbf {X} wants to perform one
of the three operations (read, edit, control) on this document.
The access control will use the following process :
\begin{itemize}
	\item Reading operation case\\
	To be authorized to perform the reading action on the document
	\dir {mydirectory/mysubdirectory/myfile}, the user \textbf {X} must be
	authorized in the description file to read each document of the
	path in their description file.
	Thus, to be allowed to read the file 
	\dir {mydirectory/mysubdirectory/myfile}, \textbf {X} must be allowed
	to read \dir {myfile} (the scenario of the description file
	of \dir {myfile} must return \textit {do\_it} with the user \textbf {X}), but
	must also be allowed to read \dir {mysubdirectory} 
	and \dir {mydirectory}.\\
	However, the owner of a document or of one of the parent directories
	of the document are allowed to perform \textbf {each action on the document}.
	So, the owners of \dir {myfile}, \dir {mysubdirectory},
	and \dir {mydirectory} are authorized to read
	 \dir {mydirectory/mysubdirectory/myfile}.

	So we can resume this with :
\begin {quote}
\begin{verbatim}
	X can read <a/b/c> 

	if			  

	(X can read <c>
	AND X can read <b>
	AND X can read <a>)
					
	OR

	(X owner of <c>
	OR X owner of <b>
	OR X owner of <a>)
\end{verbatim}			
\end {quote}

	\item Editing operation case\\
	The regulars for edit are exactly the same as read.
\begin {quote}
\begin{verbatim}
	X can edit <a/b/c> 
	
	if 
		
	(X can edit <c>
	AND X can edit <b>				
	AND X can edit <a>)
					
	OR

	(X owner of <c>
	OR X owner of <b>
	OR X owner of <a>)
\end{verbatim}			
\end {quote}

	\item Control operation case\\
	The access control before performing the control action (change rights
	or set the owner of a document) is much more restrictive.
	Only the owner of a document or the owners of one of the parent
	documents are allowed to control this document.
	Which can be resumed with :
\begin {quote}
\begin{verbatim}
	X can control <a/b/c> 

	if
					
	(X owner of <c>
	OR X owner of <b>
	OR X owner of <a>)
\end{verbatim}			
\end {quote}

\end{itemize}

\section {The shared documents actions behaviour}

The shared web feature has called for some new actions.
\begin{itemize}
	\item action D\_ADMIN\\
	Create the shared web, close it or restore it. The d\_admin action is accessible
	from the page \textbf {admin} of the list.
	\item action D\_READ\\
	Reads the document after read access control. If it's a folder, lists all the subdocuments that can
	be read. If it's a file, shows it if it is viewable and download it else.
	If the document to read contains a file named \file {index.html} or \file {index.htm} and if
	the user has no other privileges than read on all the documents inside, the read action will
	consist in displaying the index.
	The d\_read action is accessible from the page \textbf {info} of the list.
	\item action D\_CREATE\_DIR\\
	Creates a new subdirectory in a directory that can be edited. 
	The creator is the owner of the directory. The access rights are
	those of the parent directory.
	\item action D\_DESCRIBE\\
	Describes a document that can be edited.
	\item action D\_DELETE\\
	Deletes a document after edit access control. If it's a folder, it has to be empty.
	\item action D\_UPLOAD\\
	Uploads a file in a directory that can be edited.  
	\item action D\_OVERWRITE\\
	Overwrites a file if it can be edited. The new owner of the file is the one who has done
	the overwriting operation.
	\item actions D\_EDIT\_FILE and D\_SAVE\_FILE\\
	Edits a file and save it after edit access control. The new owner of the file is the one 
	who has done the saving operation. 
	\item action D\_CHANGE\_ACCESS\\
	Changes the access right to a document (read or edit) if can be controled. 
	\item action D\_SET\_OWNER\\
	Changes the owner of a directory if can be controled. The directory must be empty. The new 
	owner can be anyone but he will have to identify to perform any action on the document.

\end{itemize}

\section {The template files}
The following template files have been created for shared web:

\subsection {d\_read.tpl} 
The default page for reading a document. If it's a file, display it if it's
viewable and download it. If it's a directory, shows the listing of all
the subdocuments that can be read, and for each one, proposes the buttons 
to perform actions that are allowed. If the directory can be
edited, proposes the buttons to describe it, upload a file
inside, create a new subdirectory. If it can be edit, propose a button to
edit the access to it. 
\subsection {d\_editfile.tpl} 
The page used to edit a file. If it's a text file, permits to edit it on-line.
The description of the file can be edited in this page. The file can be replaced
by another using this page.

\subsection {d\_control.tpl}
The page to edit the access rights and the owner of a document. 

%%%%%%%%%%%%%%%%%%%%%%%%%%%%%%%%%%%%%%%%%%%%%%%%%%%%%%%%%%%%%%%%%%%%%%%%%%%%%%
% Using Sympa commands
%%%%%%%%%%%%%%%%%%%%%%%%%%%%%%%%%%%%%%%%%%%%%%%%%%%%%%%%%%%%%%%%%%%%%%%%%%%%%%

\cleardoublepage
\chapter {Bounces management}

\Sympa allows bounces (non-delivery reports) management. This
prevents list owners from receiving each bounce (1 per message
sent to a bouncing subscriber) in their own mailbox. Without
an automatic processing of bounces, list owners either become
mad or just delete them without further attention.

Bounces are received at \samplelist-owner address, which should
be sent to \file {bouncequeue} program through aliases :

\begin {quote}
\begin{verbatim}
	\samplelist-owner: "|/home/sympa/bin/bouncequeue \samplelist"
\end{verbatim}
\end {quote}

\file {bouncequeue} (see \ref{binaries}, page~\pageref{binaries}), stores bounces in a
\tildedir {sympa/spool/bounce/} spool.

Bounces are then processed by \file {bounced.pl} daemon
This daemon analyses bounces to find out what
email addresses are concerned and what kind of error was generated.
If bouncing addresses matche a subscriber's address, information 
are stored in \Sympa database (in subscriber\_table). Moreover, latest
bounce itself is archived in \dir {bounce\_path/\samplelist/email}
(where bounce\_path is defined in a \file {wwsympa.conf} parameter and
email is the user email address).

When reviewing a list, bouncing addresses are tagged as bouncing ;
you may access further information such as : first and last bounces
date, number of received bounces for the address, the last bounce itself.

Further development of \Sympa should allow automatic deletion
of bouncing addresses.


%%%%%%%%%%%%%%%%%%%%%%%%%%%%%%%%%%%%%%%%%%%%%%%%%%%%%%%%%%%%%%%%%%%%%%%%%%%%%%
% Using Sympa commands
%%%%%%%%%%%%%%%%%%%%%%%%%%%%%%%%%%%%%%%%%%%%%%%%%%%%%%%%%%%%%%%%%%%%%%%%%%%%%%

\cleardoublepage
\chapter {Using \Sympa commands}

Users interact with \Sympa, of course, when they send messages to
one of the lists, but also indirectly through administrative requests
(subscription, list of users, etc.).

This section describes administrative requests, as well as interaction
modes in the case of private and moderated lists.  Administrative
requests are messages whose body contains commands understood by
\Sympa, one per line. These commands can be indiscriminately placed
in the \rfcheader {Subject} or in the body of the message. The
\rfcheader {To} address is generally the \mailaddr {sympa{\at}domain}
alias, although it is also advisable to recognize the \mailaddr
{listserv{\at}domain} address.

Example:

\begin {quote}
\begin{verbatim}
From: pda@prism.uvsq.fr
To: sympa@cru.fr

LISTS
INFO sympa-users
REVIEW sympa-users
QUIT
\end{verbatim}
\end {quote}

Most user commands can have three-letter abbreviations (e.g. \texttt
{REV} instead of \mailcmd {REVIEW}).

\section {User commands}

\begin {itemize}
    \item  \mailcmd {HELP}

        Provides the instructions for use of \Sympa commands.  The
        result is the content of the \file {helpfile.tpl} template
        file. 

    \item  \mailcmd {INFO} \textit {listname}

        Provides the welcome message for the specified list. The
        result is the content of the \tildefile {welcome[.mime]}.

    \item  \mailcmd {LISTS}
        \label {cmd-lists}

        Provides the names of lists managed by \Sympa.  This list
        can either be generated dynamically, using \lparam {visibility}
        (see \ref {par-visibility}, page~\pageref {par-visibility})
        and \texttt {subject} list parameters (\ref {par-subject},
        page~\pageref {par-subject}) but also \tildefile
        {sympa/expl/lists.header} and \tildefile
        {sympa/expl/lists.footer}).  It can also be generated
        statically by including contents of the \tildefile
        {sympa/expl/lists} file, which must be updated manually by
        the robot \textindex {administrator}.

    \item  \mailcmd {REVIEW} \textit {listname}
        \label {cmd-review}

        Provides the parameters of the specified list (owner,
        subscription mode, etc.), as well as the addresses of
        subscribers if the run mode authorizes it. See the \lparam
        {review} parameter (\ref {par-review}, page~\pageref
        {par-review}) for the configuration file of each list,
        which controls consultation authorizations for the subscriber
        list. Since subscriber addresses can be abused by spammers,
        it is strongly advised to \textbf {only authorize owners
        to access the susbcriber list}.

    \item  \mailcmd {WHICH}
         \label {cmd-which}

        Returns the list of lists to which one is subscribed,
        as well as the configuration of his or her subscription to
        each of the lists (DIGEST, NOMAIL, SUMMARY, CONCEAL).

\item  \mailcmd {STATS} \textit {listname}
        \label {cmd-stats}

        Provides statistics for the specified list:
        number of messages received, number of messages sent,
        mega-bytes received, mega-bytes sent. This is the contents
        of the \tildefile {sympa/expl/stats} file

    \item  \mailcmd {INDEX} \textit {listname}
        \label {cmd-index}

        Provides index of archives for specified list. Access rights
        to this function are the same as for the \mailcmd {GET}
        command. 

    \item  \mailcmd {GET} \textit {listname} \textit {archive}
        \label {cmd-get}

        To retrieve archives for list (see above).  Access
        rights are the same as for the \mailcmd {REVIEW} command.
        See \lparam {review} parameter (\ref {par-review},
        page~\pageref {par-review}).

    \item  \mailcmd {LAST} \textit {listname}
        \label {cmd-last}

        To receive the last message distributed in a list (see above).  Access
        rights are the same as for the \mailcmd {GET} command.

    \item  \mailcmd {SUBSCRIBE} \textit {listname firstname name}
        \label {cmd-subscribe}

        Requests sign-up to the specified list. The \textit
        {firstname} and \textit {name} are optional. If the
        list is parameterized with a restricted subscription (see
        \lparam {subscribe} parameter, \ref {par-subscribe},
        page~\pageref {par-subscribe}), this command is sent to the
        list owner for approval.

    \item  \mailcmd {INVITE} \textit {listname user@host name}
        \label {cmd-invite}

        Invite someone to subscribe to the specified list. The 
        \textit {name} is optional. The command is similar to the
        \mailcmd {ADD} but the specified person is not added to the
        list but invited to subscribe to it accordind to the 
        \lparam {subscribe} parameter, \ref {par-subscribe},
        page~\pageref {par-subscribe}).


    \item  \mailcmd {SIGNOFF} \textit {listname} [ \textit {user@host} ]
        \label {cmd-signoff}

        Requests unsubscription from the specified list.
        \mailcmd {SIGNOFF *} means unsubscription from all lists.

    \item  \mailcmd {SET} \textit {listname} \texttt {DIGEST}
        \label {cmd-setdigest}

        Puts the subscriber in \textit {digest} mode for the \textit
        {listname} list.  Instead of receiving mail from the list
        in a normal manner, the subscriber will periodically receive
        it in a DIGEST. This DIGEST compiles a group of messages
        from the list, using multipart/digest mime format.

        The sending period for these DIGESTS is regulated by the
        list owner using the \lparam {digest} parameter (see~\ref
        {par-digest}, page~\pageref {par-digest}).  See the \mailcmd
        {SET~LISTNAME~MAIL} command (\ref {cmd-setmail}, page~\pageref
        {cmd-setmail}) and the \lparam {reception} parameter (\ref
        {par-reception}, page~\pageref {par-reception}).

    \item  \mailcmd {SET} \textit {listname} \texttt {SUMMARY}
        \label {cmd-setsummary}

        Puts the subscriber in \textit {summary} mode for the \textit
        {listname} list.  Instead of receiving mail from the list
        in a normal manner, the subscriber will periodically receive
        the list of messages. This mode is very close to the DIGEST
        reception mode but the subscriber receive only the list of messages.

        This option is available only if the digest mode is set.

    \item  \mailcmd {SET} \textit {listname} \texttt {NOMAIL}
        \label {cmd-setnomail}

        Puts subscriber in \textit {nomail} mode for the \textit
        {listname} list.  This mode is used when a subscriber wants
        to no longer receive mail from the list; nevertheless he
        or she maintains the capability of writing in the list.
        This mode therefore prevents the subscriber from unsubscribing
        and subscribing later on.  See the \mailcmd {SET~LISTNAME~MAIL}
        command (\ref {cmd-setmail}, page~\pageref {cmd-setmail}) and
        the \lparam {reception} (\ref {par-reception}, page~\pageref
        {par-reception}). 

    \item  \mailcmd {SET} \textit {listname} \texttt {MAIL}
        \label {cmd-setmail}

        Puts the subcriber in normal mode (default) for the \textit
        {listname} list.  This option is mainly used to cancel the
        \textit {nomail}, \textit {summary} or \textit {digest} modes. If the subscriber
        was in \textit {nomail} mode, he or she will again receive
        mail from the list in a normal manner.  See the \mailcmd
        {SET~LISTNAME~NOMAIL} command (\ref {cmd-setnomail},
        page~\pageref {cmd-setnomail}) and the \lparam {reception}
        parameter (\ref {par-reception}, page~\pageref {par-reception}).

    \item  \mailcmd {SET} \textit {listname} \texttt {CONCEAL}
        \label {cmd-setconceal}

        Puts the subscriber in \textit {conceal} mode for the
        \textit {listname} list.  The subscriber will then become
        invisible during \mailcmd {REVIEW} on this list. Only owners
        will see the whole subscriber list.

        See the \mailcmd {SET~LISTNAME~NOCONCEAL} command (\ref
        {cmd-setnoconceal}, page~\pageref {cmd-setnoconceal}) and
        the \lparam {visibility} parameter (\ref {par-visibility},
        page~\pageref {par-visibility}).


    \item  \mailcmd {SET} \textit {listname} \texttt {NOCONCEAL}
        \label {cmd-setnoconceal}

        Puts the subscriber in \textit {noconceal} mode (default)
        for \textit {listname} list. The subscriber will then
        become visible during \mailcmd {REVIEW} of this list. The
        \textit {conceal} mode is then cancelled.

        See \mailcmd {SET~LISTNAME~CONCEAL} command (\ref
        {cmd-setconceal}, page~\pageref {cmd-setconceal}) and
        \lparam {visibility} parameter (\ref {par-visibility},
        page~\pageref {par-visibility}).


    \item  \mailcmd {QUIT}
        \label {cmd-quit}

        Ends acceptance of commands. This can prove useful when
        the message contains additional lines, as for example in
        the case where a signature is automatically added by the
        user's program.

    \item  \mailcmd {CONFIRM} \textit {key}
        \label {cmd-confirm}

        If the \lparam {send} parameter of a list is set to \texttt
        {privatekey, publickey} or \texttt {privateorpublickey},
        messages are only distributed in the list after an
        \textindex {authentication} phase by return mail, using a
        one-time password (numeric key). For this authentication,
        the sender of the message is requested to post the ``\mailcmd
        {CONFIRM}~\textit {key}'' command to \Sympa.

    \item  \mailcmd {QUIET}

        This command is used for silent (mute) processing: no
        performance report is returned for commands prefixed with
        \mailcmd {QUIET}).

\end {itemize}

\section {Owner commands}

Some administrative requests are only available to list owner(s).
They are indispensable for all procedures in limited access mode,
and to perform requests in place of users.
These requests are:

\begin {itemize}
    \item \mailcmd {ADD} \textit {listname user@host firstname name}
        \label {cmd-add}

        Add command similar to \mailcmd {SUBSCRIBE}

    \item \mailcmd {DELETE} \textit {listname user@host}
        \label {cmd-delete}

        Delete command similar to \mailcmd {SIGNOFF}

    \item \mailcmd {REMIND} \textit {listname} or \mailcmd {REMIND} \textit {*}
	\label {cmd-remind}

        \mailcmd {REMIND} is used usually by list owner in order to send
        an individual service message to each subscriber. This
        message is make by parsing the remind.tpl file.

        \mailcmd {REMIND} is used to send to each subscriber of any list a single
        message with a summary of his/her subscriptions. In this case the 
        message sent is built by parsing the global\_remind.tpl file.
        Freach list  sympa test if the list is setup as hidden 
	to each subscriber (parameter lparam {visibility}). By default this 
	command usage is limited to listmasters. 
	Processing may take a lot of time !
	
    \item  \mailcmd {EXPIRE}
        \label {cmd-expire}

        % [pda] : huh ???
        % [sa] on touche pas la doc avant d'avoir revu le code
        \textit {listname}
        \textit {age (in days)}
        \textit {deadline (in days)}
        (listname) (age (in days)) (deadline (in days))
        \textit {explanatory text to be sent to concerned subscribers}


        This command activates an \textindex {expiration} process
        for former subscribers of the designated list. Subscribers
        for which no procedures have been enabled for more than
        \textit {age} days receive the explanatory text appended
        to the \mailcmd {EXPIRE} command. This text, which must be
        adapted by the list owner for each subscriber population,
        should explain to the people receiving this message that
        they can update their subscription date so as to not be
        deleted from the subscriber list, within a deadline of
        \textit {deadline} days.

        Past this deadline, the initiator of the \mailcmd {EXPIRE}
        command receives the list of persons who have not confirmed
        their subscription.  It is up to the initiator to send
        \Sympa the corresponding \mailcmd {DELETE} commands.

        Any operation updating the subscription date of an address
        serves as confirmation of subscription. This is also the
        case for \mailcmd {SET} option selecting commands and for
        the \mailcmd {SUBSCRIBE} subscription command itself. The fact
        of sending a message to the list also updates the subscription
        date.

        The explanatory message should contain at least 20 words;
        it is possible to delimit it by the word \mailcmd {QUIT},
        in particular in order not to include a signature, which
        would systematically end the command message.

        A single expiration process can be activated at any given
        time for a given list. The \mailcmd {EXPIRE} command
        systematically gives rise to \textindex {authentication}
        by return mail.  The \mailcmd {EXPIRE} command has \textbf
        {no effect on the subscriber list}.

    \item  \mailcmd {EXPIREINDEX} \textit {listname}
       \label {cmd-expireindex}

       Makes it possible, at any time, for an expiration process
       activated using an \mailcmd {EXPIRE} command to receive the
       list of addresses for which no enabling has been received.

    \item  \mailcmd {EXPIREDEL} \textit {listname}
       \label {cmd-expiredel}

       Deletion of a process activated using the \mailcmd {EXPIRE}
       command.  The \mailcmd {EXPIREDEL} command has no effect on
       subscribers, but it possible to activate a new expiration
       process with new deadlines.

\end {itemize}

As above, these commands can be prefixed with \mailcmd {QUIET} to
indicate processing without acknowledgment of receipt.


\section {Moderator commands}
    \label {moderation}

\Sympa only distributes messages enabled by one of the moderators
(editor) of the list in the moderated list. Moderators have several
methods for enabling message distribution, depending on the \lparam
{send} list parameter (\ref {par-send}, page~\pageref {par-send}).

\begin {itemize}
    \item  \mailcmd {DISTRIBUTE} \textit {listname} \textit {key}
        \label {cmd-distribute}

        If the \lparam {send} parameter of a list is set to \texttt
        {editorkey} or \texttt {editorkeyonly}, each message queued
        for \textindex {moderation} is stored in a spool (see~\ref
        {cf:queuemod}, page~\pageref {cf:queuemod}), and a key is
        associated to it.

        The \textindex {moderator} must use this command to enable
        message distribution.

    \item  \mailcmd {REJECT} \textit {listname} \textit {key}
        \label {cmd-reject}

        The message with the \textit {key} key is deleted from the
        moderation \textindex {spool} of the \textit {listname}
        list.

    \item  \mailcmd {MODINDEX} \textit {listname}
        \label {cmd-modindex}

        This command sends back the list of messages queued for
        moderation for the \textit {listname} list.

        The result is presented in the form of an index, which
        supplies, for each message, its sending date, its sender,
        its size, and the key associated to it, as well as all
        messages in the form of a digest.

\end {itemize}

% [pda] : c'est int�ressant, mais c'est en fran�ais et ce n'est pas dans cette doc. Je pense que le mieux serait de le traduire en anglais et de l'inclure dans cette doc
% [sa] : OK a faire laisse le commentaire.
% [pda] : et de renvoyer l'URL ci-dessus vers la page HTML correspondante dans la doc
See also the
\htmladdnormallinkfoot {advice documents for the moderation} {http://listes.cru.fr/admin/moderation.html}.

%%%%%%%%%%%%%%%%%%%%%%%%%%%%%%%%%%%%%%%%%%%%%%%%%%%%%%%%%%%%%%%%%%%%%%%%%%%%%%
% Appendices
%%%%%%%%%%%%%%%%%%%%%%%%%%%%%%%%%%%%%%%%%%%%%%%%%%%%%%%%%%%%%%%%%%%%%%%%%%%%%%


%%%%%%%%%%%%%%%%%%%%%%%%%%%%%%%%%%%%%%%%%%%%%%%%%%%%%%%%%%%%%%%%%%%%%%%%%%%%%%
% Index
%%%%%%%%%%%%%%%%%%%%%%%%%%%%%%%%%%%%%%%%%%%%%%%%%%%%%%%%%%%%%%%%%%%%%%%%%%%%%%

\cleardoublepage
\printindex

\end {document}
